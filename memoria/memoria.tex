% **************************************************************************************************************
% A Classic Thesis Style
% An Homage to The Elements of Typographic Style
%
% Copyright (C) 2015 André Miede http://www.miede.de
%
% If you like the style then I would appreciate a postcard. My address 
% can be found in the file ClassicThesis.pdf. A collection of the 
% postcards I received so far is available online at 
% http://postcards.miede.de
%
% License:
% This program is free software; you can redistribute it and/or modify
% it under the terms of the GNU General Public License as published by
% the Free Software Foundation; either version 2 of the License, or
% (at your option) any later version.
%
% This program is distributed in the hope that it will be useful,
% but WITHOUT ANY WARRANTY; without even the implied warranty of
% MERCHANTABILITY or FITNESS FOR A PARTICULAR PURPOSE.  See the
% GNU General Public License for more details.
%
% You should have received a copy of the GNU General Public License
% along with this program; see the file COPYING.  If not, write to
% the Free Software Foundation, Inc., 59 Temple Place - Suite 330,
% Boston, MA 02111-1307, USA.
%
% **************************************************************************************************************
\RequirePackage{fix-cm} % fix some latex issues see: http://texdoc.net/texmf-dist/doc/latex/base/fixltx2e.pdf
\documentclass[ oneside,openany,titlepage,numbers=noenddot,headinclude,%1headlines,% letterpaper a4paper
                footinclude=true,cleardoublepage=empty,abstractoff, % <--- obsolete, remove (todo)
                BCOR=5mm,paper=a4,fontsize=11pt,%11pt,a4paper,%
                spanish,american%
                ]{scrreprt}

%********************************************************************
% Note: Make all your adjustments in here
%*******************************************************
\input{classicthesis-config}
%% Stuff from Pandoc
\providecommand{\tightlist}{%
  \setlength{\itemsep}{0pt}\setlength{\parskip}{0pt}}

\setlength{\parskip}{1mm}
%\setlength{\mathindent}{0mm}

%% Algorithms
\usepackage{algorithm}
\usepackage{algorithmic}
\renewcommand{\algorithmicrequire}{\textbf{Entrada:}}
\renewcommand{\algorithmicensure}{\textbf{Salida:}}
\makeatletter
\renewcommand{\ALG@name}{Algoritmo}
\renewcommand{\listalgorithmname}{Lista de \ALG@name s}
\makeatother

%% Tikz for neural network diagrams
\usepackage{tikz}

%% Theorem environments
\usepackage{aliascnt}
\def\NewTheorem#1#2{%
  \newaliascnt{#1}{theorem}
  \newtheorem{#1}[#1]{#2}
  \aliascntresetthe{#1}
  \expandafter\def\csname #1autorefname\endcsname{#2}
}
\usepackage{amsthm}
% \newtheorem{satz}{Satz}[chapter]
% \newtheorem*{satz*}{Satz}
% \newtheorem{lemma}[satz]{Lemma}
% \newtheorem{corollar}[satz]{Korollar} 
% \newcommand{\satzautorefname}{Satz}
\newtheorem{theorem}{Teorema}[chapter]
\NewTheorem{lemma}{Lema}
\NewTheorem{prop}{Proposición}
\NewTheorem{cor}{Corolario}
%\newtheorem{lemma}[theorem]{Lema}
%\newtheorem{prop}[theorem]{Proposición}
%\newtheorem{cor}[theorem]{Corolario}

\theoremstyle{definition}
\newtheorem{definition}{Definición}[chapter]
\newtheorem{example}{Ejemplo}[chapter]
\newtheorem{exca}{Ejercicio}[chapter]

\theoremstyle{remark}
\newtheorem{remark}{Observación}[chapter]

% Autoref commands
%\newcommand{\corollaryautorefname}{corolario}
%\newcommand{\corautorefname}{corolario}
%\newcommand{\propautorefname}{proposición}
%\newcommand{\propositionautorefname}{proposición}
%\newcommand{\lemmaautorefname}{lema}
\newcommand*{\definitionautorefname}{definición}
\newcommand*{\exampleautorefname}{ejemplo}
\newcommand*{\remarkautorefname}{observación}
\newcommand*{\algorithmautorefname}{algoritmo}
\def\theoremautorefname{teorema}
\def\sectionautorefname{sección}

%\numberwithin{equation}{section}

% Replacing environments by pairs of commands for use in Markdown
\newcommand{\defineb}{\begin{definition}}
\newcommand{\definee}{\end{definition}}
\newcommand{\theob}{\begin{theorem}}
\newcommand{\theoe}{\end{theorem}}
\newcommand{\lemmab}{\begin{lemma}}
\newcommand{\lemmae}{\end{lemma}}
\newcommand{\propb}{\begin{prop}}
\newcommand{\prope}{\end{prop}}
\newcommand{\remb}{\begin{remark}}
\newcommand{\reme}{\end{remark}}
\newcommand{\proofb}{\begin{proof}}
\newcommand{\proofe}{\end{proof}}
\newcommand{\exampleb}{\begin{example}}
\newcommand{\examplee}{\end{example}}
\newcommand{\corb}{\begin{cor}}
\newcommand{\core}{\end{cor}}

% hyperref
%\PassOptionsToPackage{xetex,hyperfootnotes=false,pdfpagelabels}{hyperref}
%    \usepackage{hyperref}

%% ATAJOS
\newcommand{\RR}{\mathbb{R}}
\newcommand{\NN}{\mathbb{N}}
\newcommand{\ZZ}{\mathbb{Z}}
\newcommand{\KK}{\mathbb{K}}
\newcommand{\LL}{\mathcal{L}}

% Distribuciones de probabilidad
\newcommand{\PN}{\mathcal{N}}

% Traspuesta
\newcommand{\Tr}[1]{#1^{\mathrm{T}}}
% Norma
\newcommand{\norm}[1]{\left\lVert#1\right\rVert}

%\newcommand{\dim}{\mathrm{dim}}
\newcommand{\E}{\mathrm{E}}
\let\Pr\relax
\newcommand{\Pr}[1]{\mathrm{P}\left[#1\right]}
\newcommand{\Var}{\mathrm{Var}}
\newcommand{\Ker}{\mathrm{Ker}}
\newcommand{\Mid}{\mid\mid}

\newcommand{\asconv}{\overset{cs}{\rightarrow}}
\newcommand{\pconv}{\overset{P}{\rightarrow}}
\newcommand{\softmax}[1]{\mathrm{softmax}(#1)}

\newcommand{\note}[1]{\paragraph{Nota}#1}

%********************************************************************
% Bibliographies
%*******************************************************
\addbibresource{bibliography.bib}

%********************************************************************
% Hyphenation
%*******************************************************
%\hyphenation{put special hyphenation here}

% ********************************************************************
% GO!GO!GO! MOVE IT!
%*******************************************************
\begin{document}
\frenchspacing
\raggedbottom
\selectlanguage{spanish} % american ngerman
%\renewcommand*{\bibname}{new name}
%\setbibpreamble{}
\pagenumbering{roman}
\pagestyle{plain}
%********************************************************************
% Frontmatter
%*******************************************************
%\include{FrontBackmatter/DirtyTitlepage}
%*******************************************************
% Titlepage
%*******************************************************
\begin{titlepage}
    % if you want the titlepage to be centered, uncomment and fine-tune the line below (KOMA classes environment)
    \begin{addmargin}[-3.45cm]{-3cm}
    \begin{center}
        \large  

        \hfill

        \includegraphics[width=6cm]{images/ugrmarca} \\ \medskip

        \vfill

        \begingroup
            \color{TealBlue}\spacedallcaps{\myTitle} \\ \bigskip
        \endgroup

        \spacedlowsmallcaps{\mySubtitle}

        \vfill

        \myWork \\ 
        \myDegree \\ \bigskip
        \textbf{Autor} \\
        \myName \\ \medskip
        \textbf{Tutor} \\
        \myProf \\ \bigskip
        %\myDepartment \\                            
        \spacedlowsmallcaps{\myFaculty} \\ \medskip
        \spacedlowsmallcaps{\myOtherFacultyA} \\
        \spacedlowsmallcaps{\myOtherFacultyB} \\ \bigskip
        %\spacedlowsmallcaps{\myUni} \\ 

        \myLocation, \myTime %\ -- \myVersion

        \vfill                      

    \end{center}  
  \end{addmargin}       
\end{titlepage}   
\thispagestyle{empty}

\hfill

\vfill

\pdfbookmark[1]{Licencia}{Licencia}
\textbf{Licencia} %\ccbysa
\par\vspace*{\dimexpr-\parskip-\baselineskip+6pt}
\noindent\rule{\textwidth}{0.5pt}

Esta obra está sujeta a la \hyperlink{https://creativecommons.org/licenses/by-sa/4.0/legalcode}{licencia Reconocimiento-CompartirIgual 4.0 Internacional de Creative Commons}.

La licencia permite:
\begin{itemize}
	\item[] \textbf{Compartir} --- copiar y redistribuir el material en cualquier medio o formato.
	\item[] \textbf{Adaptar} --- remezclar, transformar y crear a partir del material
para cualquier finalidad, incluso comercial.
\end{itemize}

Bajo las condiciones siguientes:
\begin{itemize}
	\item[] \textbf{Reconocimiento} --- Debe reconocer adecuadamente la autoría, proporcionar un enlace a la licencia e indicar si se han realizado cambios. Puede hacerlo de cualquier manera razonable, pero no de una manera que sugiera que tiene el apoyo del licenciador o lo recibe por el uso que hace. 
	\item[] \textbf{CompartirIgual} --- Si remezcla, transforma o crea a partir del material, deberá difundir sus contribuciones bajo la misma licencia que el original.
	\item[] \textbf{No hay restricciones adicionales} --- No puede aplicar términos legales o medidas tecnológicas que legalmente restrinjan realizar aquello que la licencia permite.
\end{itemize}
\clearpage

%\cleardoublepage\include{FrontBackmatter/Dedication}
%\cleardoublepage\include{FrontBackmatter/Foreword}
\cleardoublepage%*******************************************************
% Abstract
%*******************************************************
%\renewcommand{\abstractname}{Abstract}
\pdfbookmark[1]{Resumen}{Resumen}

\chapter*{Resumen}

En este trabajo se analizan desde una perspectiva teórica las técnicas basadas en redes neuronales profundas que permiten abordar el problema de reducción de la dimensionalidad y se explica el software desarrollado, que permite el uso de dichas técnicas y la generación de visualizaciones sobre ellas, bien mediante programación o bien a través de una interfaz gráfica de usuario web.

Primero se introducen conceptos matemáticos que ayudan a comprender los algoritmos y modelos que fundamentan estas redes profundas, haciendo hincapié en resultados teóricos que dan ideas acerca del problema que se va a abordar. Posteriormente, se describen los algoritmos y que realizan el aprendizaje sobre dichas estructuras, y las arquitecturas relevantes que tratan este problema. Por último, se documenta el software implementado y se muestran ejemplos de su uso.

\paragraph{Palabras clave} Deep Learning, redes neuronales, reducción de dimensionalidad, clasificación, aprendizaje no supervisado, probabilidad, teoría de la información.


\begin{otherlanguage}{american}
\pdfbookmark[1]{Abstract}{Abstract}
\chapter*{Abstract}

This work studies, from a theoretical perspective, techniques based on deep neural networks that tackle the high dimensionality problem. It also explains the piece of software developed for this project, which allows the use of said techniques and includes resources for visualization, offering both a programming interface and a web-based graphic user interface.

\section*{Description of the addressed problem}

The current trend in data collection from diverse sources for its subsequent processing implies the need for powerful learning algorithms as well as high computation abilities. One of the most common tasks is classification, where an algorithm attempts to predict one or several labels tied to data samples, having previously learned from already classified examples.

Classification algorithms often suffer some performance loss when trained against high dimensional data sets, a phenomenon known as the \emph{curse of dimensionality}. From the very varied approaches to tackle this problem, this work focuses on dimensionality reduction via unsupervised deep neural networks. Neural networks generalize the perceptron, and deep networks are an extension of this concept. It was defined decades ago but has made a comeback thanks to the progress on high performance computing and efficient training algorithms.

Techniques of this kind can be found in software for languages such as Python or C++, whereas the data-oriented language R lacks a library offering easy access to unsupervised deep neural networks. The software implemented in this project, Ruta, aims to resolve this absence. Furthermore, a companion tool named Rutavis adds visualization mechanisms and a web-based graphic user interface.

The documentation is divided into two distinct parts. One offers a mathematical framework which serves as a basis for the concepts needed to describe the Deep Learning techniques related to this work. The areas of mathematics used are probability theory, information theory and tensor algebra. The other main part describes the common machine learning notions, problems and algorithms that also apply to Deep Learning and continues with the specification of the techniques used in unsupervised deep neural networks. It finishes by describing the design and implementation of both pieces of software, Ruta and Rutavis.

The following sections constitute a summary of this documentation.

\section*{Mathematical foundations of Deep Learning}

Deep Learning finds its roots in several areas of mathematics. Especially, there are important concepts to be studied in probability theory, information theory and tensor algebra.

The needed definitions from probability theory are very basic but essential to the rest of the text, so special attention is given to rigurously compile them. The ideas of probability distribution, conditional distribution, independence and the main moments (expectation and variance) are explained. As a preliminary result, the theorem of the continuous mapping is formulated. It is then used to prove one of the main theoretical results of this work, a theorem describing the \emph{curse of dimensionality} that motivates the task known as dimensionality reduction. As a consequence it is deduced that, as the dimensionality of a data set increases, the difference between the nearest neighbor and the farthest one becomes insignificant.

The next chapter introduces the convenient notions around entropy. The entropy of a random variable is defined and several properties are deduced. Then, measures of the information involved among two variables are presented as the joint entropy and conditional entropy. These are used to infer the chain rule of entropy. Later, two important concepts are defined: the Kullback-Leibler divergence and cross entropy, which will be applied later to the construction of objective functions in deep neural networks. Some properties and alternative expressions of these concepts are verified. Lastly, Jensen's inequality, which is a well-known result among the mathematical field, is used to prove the information inequality and its consequences. Intuitively, it is deduced that probabilistic models assumed around sampled data need to be as close as the true probability distribution as possible in order to be able to optimize the compression of data.

After that, a chapter is dedicated to concepts around the notion of tensor. The mathematical foundations of tensors rely on multilinear algebra, which is explained starting from linear functionals and dual spaces, then introducing the mechanics of the change of coordinates within these spaces. The second dual is shown to be trivially isomorphic to the origin vector space. These results allow the definition of multilinear mappings and, in particular, multilinear functionals or tensors. The tensor product of functionals is introduced and later used to define a base for the tensor product of vector spaces. The distinction between covariant and contravariant tensors is explained. Finally, the use of tensors in machine learning is shown to be a distant application of the previous definitions.

\section*{Algorithms and structures in Deep Learning}

Deep Learning is considered a branch of machine learning. Thus, a notable amount of theoretical content and algorithms can be applied to the problem studied. However, many of the latest developments on Deep Learning are novel and deserve a thorough study.

The essential notions of learning and learners are introduced and several common learning tasks are enumerated. The two main types of learning, supervised and unsupervised, are distinguished. Some examples are also provided. Later, the classification problem is presented, and a theoretical formulation with a simple notation is introduced, which relates to the field of PAC learning. Different types of classification problems are enumerated as well. The structure of the space where the data belongs is lightly discussed afterwards.

The main problem addressed in this work, the dimensionality reduction task, is described and connected to the theoretical results proved before. Several means of action that tackle this problem are explained, and the approach studied in the work is highlighted. There is also a breakdown of the process referred to as feature extraction, and some mechanisms to build new features are mentioned. Lastly, one of the most common optimization algorithms in machine learning, gradient descent, is thoroughly described. It will serve as a basis for the optimization methods in Deep Learning.

Afterwards, the chapter dedicated to Deep Learning begins by outlining the common structure used in many of its applications, the deep feedforward neural network. A mathematical notation is introduced and the biological inspiration behind the artificial neuron is discussed. Later, an introduction to the way cost functions are derived from the selected probabilistic model is made. Different types of output units or neurons are explained, as well as the cost function they determine. Especially, the most commonly found ones are described: linear, sigmoid and softmax units. This explanation is extended to all hidden units in a feedforward network by adding rectified linear units and their variants, as well as the hyperbolic tangent and other activation functions.

The training process of a deep neural network requires some special techniques, known as forward propagation and backward propagation. These algorithms are motivated and thoroughly depicted, and a small example on the computation of gradients is made. The algorithms derived from gradient descent, specific for training deep networks, analyzed in this work are stochastic gradient descent and its variants AdaGrad, RMSProp and Adam. These allow to optimize the cost functions via several iterations where a minibatch of data is propagated through the network and gradients are computed.

Deep neural networks can be designed in a specific means to be trained in an unsupervised fashion. The main architectures that allow this kind of training are restricted Boltzmann machines and autoencoders. These are characterized and illustrated in the present work, paying special attention to some variants of the autoencoder and their properties: undercomplete, sparse, denoising and contractive autoencoders are all specified in quite detail. Finally, a special training procedure for autoencoders involving a stack of restricted Boltzmann machines is explained.

\section*{Software implementation: the Ruta package}

During the realization of this project, two pieces of software have been designed and implemented. One of them, named Ruta, gives uncomplicated access to unsupervised deep neural networks, from building their architecture to their training and evaluation. The second one is a complementary project called Rutavis, which allows to generate graphical representations of the models trained with Ruta.

First, an introduction to the R language is offered, and simple instructions for its installation are provided. There is also a general description of the language and the object orientations it admits. Afterwards, a neural network library called MXNet is introduced. It implements the necessary operations and algorithms needed to build and train deep architectures. Instructions for its installation are provided as well as a specification of the programming mechanic it offers, including a complete example of the training and prediction of a simple neural network oriented to regression.

The Ruta software is presented and motivated. It is mentioned that an early prototype was previously introduced at a national conference. Later, there is an exhaustive description of its structure and functionality. The main objects used to work with this package abstract the notions of learning tasks, learners and trained models. These objects are defined and exemplified making use of the Iris data set and a simple deep autoencoder.

The Rutavis package is also outlined via the same main points, its structure and functionality. Some example plots are shown and screenshots of an use case with the web-based graphic user interface are offered as well.

The document ends with some conclusions and sketches some future work.

\paragraph{Keywords} Deep Learning, neural networks, dimensionality reduction, classification, unsupervised learning, probability, information theory.

\end{otherlanguage}
%\cleardoublepage\include{FrontBackmatter/Publications}
%\cleardoublepage%*******************************************************
% Acknowledgments
%*******************************************************
\pdfbookmark[1]{Agradecimientos}{agradecimientos}

% \begin{flushright}{\slshape    
%     We have seen that computer programming is an art, \\ 
%     because it applies accumulated knowledge to the world, \\ 
%     because it requires skill and ingenuity, and especially \\
%     because it produces objects of beauty.} \\ \medskip
%     --- \defcitealias{knuth:1974}{Donald E. Knuth}\citetalias{knuth:1974} \citep{knuth:1974}
% \end{flushright}



% \bigskip

\begingroup
\let\clearpage\relax
\let\cleardoublepage\relax
\let\cleardoublepage\relax
\chapter*{Agradecimientos}

Me gustaría agradecer a la Universidad de Granada, a la Facultad y la Escuela y en concreto a sus docentes por brindarme la oportunidad de formarme. Especialmente a mi tutor, Francisco Herrera, por su apoyo a lo largo de este proyecto.

Muchas gracias a mi familia por su constante ayuda a lo largo de estos años, y a mis amistades que de una u otra forma me han animado a seguir adelante y de las cuales he aprendido tanto.

\endgroup

\pagestyle{scrheadings}
\cleardoublepage\include{meta/Contents}
%********************************************************************
% Mainmatter
%*******************************************************
\cleardoublepage\pagenumbering{arabic}
%\setcounter{page}{90}
% use \cleardoublepage here to avoid problems with pdfbookmark
\cleardoublepage

\part{Introducción}

\chapter{Descripción}
\begin{itemize}
\tightlist
\item
  Objetivo del trabajo
\item
  Introducción del problema
\item
  Idea general del contenido
\item
  Organización del trabajo
\item
  Nota sobre documentación y software libre
\end{itemize}


\chapter{Objetivos}
\begin{itemize}
\tightlist
\item
  Recopilación y estudio bibliográfico de los fundamentos matemáticos
  del Deep Learning, con especial enfoque en la Teoría de la Información
  y Teoría de la Probabilidad
\item
  Diseño, desarrollo e implementación de un software que recopile las
  técnicas de Deep Learning orientadas a reducción de dimensionalidad
  más relevantes e incluya visualizaciones y facilidades para el
  análisis
\item
  Análisis experimental de las técnicas utilizadas y relación con los
  resultados visuales generados por la herramienta software
\end{itemize}


\part{Matemáticas}

\chapter{Teoría de la Información}
\input{chapters/teoria-informacion.tex}

\chapter{Álgebra Tensorial}
El álgebra tensorial es una rama del álgebra que extiende los conceptos del álgebra lineal. Además, el concepto de tensor se traslada a la implementación de técnicas de aprendizaje automático, puesto que la aplicación que se realiza de ellos permite hacer más eficientes los cómputos de operaciones respecto al uso exclusivo de matrices.
La fuente principal de este capítulo es \textcite[capítulo 8]{treil2013}.

\section{Espacios duales}\label{espacios-duales}

En esta sección introducimos los conceptos esenciales sobre duales de espacios vectoriales, necesarios para llegar a la definición de tensor.

\subsection{Funcionales lineales y el espacio
dual}\label{funcionales-lineales-y-el-espacio-dual}

\defineb
Un \emph{funcional lineal} en un espacio vectorial finito dimensional
\(V\) sobre un cuerpo \(\KK\) es una aplicación lineal
\(L:V\rightarrow \KK\). \definee
\defineb
El \emph{espacio dual} de un espacio vectorial finito dimensional
\(V\) es \(V^*=\{L:V\rightarrow \KK\text{ lineal}\}=\mathcal{L}(V, \KK)\).
\definee

\exampleb
Sea \(V=\mathbb R^n\) y consideramos
\((\mathbb R^n)^*=\left\{L:\mathbb R^n\rightarrow\mathbb R\text{ lineal}\right\}\).
Sabemos que toda aplicación lineal de \(\mathbb{R}^{n}\) en
\(\mathbb{R}^{m}\) se expresa, fijada una base, como una matriz
\(m\times n\), luego identificamos \((\mathbb{R}^{n})^{*}\) con matrices
\(1\times n\) (en la base usual). Evidentemente hay un isomorfismo entre
este conjunto y \(\mathbb{R}^{n}\):

\begin{align*}
  (\mathbb{R}^{n})^{*} \cong \mathcal{M}_{1\times n}(\mathbb{R})&\cong \mathbb{R}^{n} \\
  (m_{1} \dots m_{n}) &\mapsto (m_{1}, \dots m_{n})
\end{align*}

Este hecho se generaliza para cualquier cuerpo \(\KK\):
\((\KK^{n})^{*}\cong \KK^{n}\). \examplee

\subsubsection{Cambio de coordenadas}\label{cambio-de-coordenadas}

Sea \(V\) \(\KK\)-espacio vectorial, sean
\(A=\left\{a_{1}, \dots a_{n}\right\}, B=\left\{b_{1}, \dots b_{n}\right\}\)
bases de \(V\) donde \(n = \dim_{\KK}V\).

Introducimos la siguiente notación: dada una base \(B\) de \(V\), \(B'\)
de \(W\), \(L\in\mathcal L(V, W)\), notaremos \([L]_{B', B}\) a la
expresión matricial de \(L\) en las bases \(B, B'\). Si \(L\in V^{*}\)
notamos \([L]_{B}\).

Sabemos que la expresión de \(L\in V^{*}\) en la base \(B\) viene dada
por su imagen por los vectores de la base, y el cambio de coordenadas es
\([L]_B=[L]_A[I]_{A,B}\).

Recordamos también que el cambio de base de \(v\in V\) se realiza
mediante \([B]_B=[I]_{B,A}[V]_A\) y que \([I]_{B,A}=[I]_{A,B}^{-1}\).
Así, llamando \(S=[I]_{B,A}\) observamos que el cambio de base de los
vectores asociados a las filas \([L]_B, [L]_A\) es:

\[
  [L]_{B}^t=(S^{-1})^t[L]_{A}^t
\]

\begin{prop}
Dado $V$ espacio vectorial, si $S$ es la matriz de cambio de base de $A$ a $B$ entonces la matriz de cambio de base de $V^{*}$ es $(S^{-1})^t$.
\end{prop}

\begin{lemma}
  \label{lemma:vectorcero}
  Sea $v\in V$. Si $L(v)=0\forall L\in V^{*}$ entonces $v=0$. Como consecuencia, si $L(v_1)=L(v_2)\forall L\in V^{*}$ entonces $v_1=v_2$.
  
  \begin{proof}
    Sea $B$ base de $V$. Entonces $L(v)=[L]_B[v]_B$. Basta tomar $L_k=[0, \dots 0, \overset{(k)}{1}, 0, \dots 0]$ y comprobar que en ese caso $L_k[v]_B=0$ implica que la $k$-ésima coordenada de $[v]_B$ es 0. Repitiendo el mismo paso para cada $k$ tenemos que $v=0$.
  \end{proof}
\end{lemma}

\subsection{El segundo dual}\label{el-segundo-dual}

Puesto que \(V^{*}\) es un espacio vectorial, podemos considerar también
su dual, que notaremos \(V^{**}\). Comprobaremos que, de hecho,
\(V^{**}\) es isomorfo a \(V\) de una forma natural. Dado \(v\in V\)
podemos tomar \(L_{v}\in V^{**}\) dado por
\(L_{v}(f) = f(v)\forall f\in V^{*}\). Así, podemos construir una
aplicación del espacio \(V\) en su segundo dual,
\(T:V\rightarrow V^{**}\), dada de forma natural por
\(Tv=L_v\forall v\in V\).

Para ver que \(T\) es un isomorfismo, observamos que las dimensiones de
los espacios coinciden: \(\dim V^{**}=\dim V^{*}=\dim V\). Por tanto,
bastará con ver que \(T\) es inyectivo: veamos para ello que
\(\Ker T=\{0\}\). Dado \(v\in \Ker T\), tenemos que
\(\forall f\in V^{*} f(v)=L_v(f)=T(v)(f)=0\). Por el
\autoref{lemma:vectorcero}, se tiene que \(v=0\).

Nótese que el isomorfismo \(T\) no depende de la elección de una base en
\(V\).

\section{Funciones multilineales.
Tensores}\label{funciones-multilineales.-tensores}

\defineb

Sean \(V_1,\dots,V_p,V\) espacios vectoriales sobre un cuerpo \(\KK\).
Una \emph{aplicación multilineal} (\(p\)-lineal) con valores en V es una
función \(F:V_1\times \dots\times V_p\rightarrow V\), lineal en cada
variable. Es decir, para cada \(k\in \{1,\dots, p\}\) y fijado
\((v_1,\dots,v_{k-1},0,v_{k+1},\dots,v_p)\in V_1\times \dots\times V_p\),
se tiene que la aplicación que lleva
\(v_k\mapsto F(v_1,\dots,v_{k-1},v_k,v_{k+1},\dots,v_p)\) es lineal.

Notamos por \(\LL(V_1,\dots,V_p;V)\) a la familia de todas las
aplicaciones \(p\)-lineales de \(V_1\times \dots\times V_p\) en \(V\).

\definee

\defineb

Un \emph{tensor} o \emph{funcional multilineal} es una aplicación
multilineal con codominio \(\KK\),
\(F:V_1\times \dots\times V_p\rightarrow \KK\). La cantidad \(p\) se
denomina el \emph{rango} o \emph{valencia} del tensor.

\definee

En particular, un tensor de rango 1 es un funcional lineal, y un
tensor de rango 2 es una forma bilineal.

\exampleb

Sean \(V_1,\dots,V_p\) \(\KK\)-espacios vectoriales y sean
\(f_1\in V_1^{*},\dots f_p\in V_p^{*}\) funcionales lineales. Definimos
el funcional multilineal \(F:V_1\times\dots\times V_p\rightarrow\KK\)
dado por
\[F(v_1,\dots,v_p)=f_1(v_1)f_2(v_2)\dots f_p(v_p),\ v_i\in V_i,\ k=1,2,\dots,p.\]

El funcional \(F\) se denomina \emph{producto tensorial} de los
funcionales \(f_i\) y lo notamos
\(F=f_1\otimes f_2\otimes\dots\otimes f_p\).

\examplee

\remb

El conjunto de las aplicaciones multilineales es un \(\KK\)-espacio
vectorial, mediante las siguientes operaciones de suma y producto por
escalar: sean \(F_1, F_2\in \LL(V_1,\dots,V_p;V),\alpha\in\KK\)

\begin{align*}
(F_1+F_2)(v_1,\dots,v_p)&=F_1(v_1,\dots,v_p)+F_2(v_1,\dots,v_p),\\
(\alpha F_1)(v_1,\dots,v_p)&=\alpha F_1(v_1,\dots,v_p).
\end{align*}

\reme

\propb
\label{prop:base-funcionales} Sean \(V_1,\dots,V_p\) \(\KK\)-espacios
vectoriales con bases \(B^{(1)},\dots,B^{(p)}\) respectivamente. Notamos
\(b_i^{(k)}\) al \(i\)-ésimo elemento de la base \(B^{(k)}\).

Para cada \(k\in\{1,\dots,p\}\) y para cada \(i\in\{1,\dots,\dim V_k\}\)
sea \(f_{i}^{(k)}\) el funcional lineal de \(V_k^{*}\) definido por

\begin{align*}
    f_{i}^{(k)}(b_i^{(k)})&=1\\
    f_{i}^{(k)}(b_j^{(k)})&=0,\ j\neq i.
\end{align*}

La familia
\[B=\left\{f_{i_1}^{(1)}\otimes\dots\otimes f_{i_p}^{(p)},\ 1\leq i_k\leq\dim{V_k},\ k\in\{1,\dots,p\}\right\}\]
es una base del espacio \(\LL(V_1,\dots, V_p;\KK)\).

En particular,
\[\dim\LL(V_1,\dots, V_p;\KK)=(\dim V_1)\dots (\dim V_p).\]

\proofb

Dada \(F\in \LL(V_1,\dots, V_p;\KK)\) queremos expresarla de forma única
en función de los elementos de la familia \(B\), es decir, buscamos
coeficientes \(\alpha_{i_1,i_2,\dots,i_p}\in\KK\) tales que

\begin{equation}
  \label{eq:basetensor}
  F=\sum\limits_{i_k\in\{1,\dots,\dim V_k\}} \alpha_{i_1,i_2,\dots,i_p}f_{i_1}^{(1)}\otimes\dots\otimes f_{i_p}^{(p)}\ .
\end{equation}

Por la definición de los funcionales, se tiene que
\begin{equation}
  \label{eq:evalbasetensor}
  f_{i_1}^{(1)}\otimes\dots\otimes f_{i_p}^{(p)}\left(b_{j_1}^{(1)},\dots,b_{j_p}^{(p)}\right)=1\Leftrightarrow i_1=j_1,\dots,i_p=j_p
\end{equation}
y, en otro caso,
\[f_{i_1}^{(1)}\otimes\dots\otimes f_{i_p}^{(p)}\left(b_{j_1}^{(1)},\dots,b_{j_p}^{(p)}\right)=0\ .\]

Evaluando ahora \(F\) \eqref{eq:basetensor} en
\(b_{i_1}^{(1)},\dots,b_{i_p}^{(p)}\):

\begin{equation*}
F\left(b_{i_1}^{(1)},\dots,b_{i_p}^{(p)}\right)=\alpha_{i_1,\dots,i_p}
\end{equation*}
lo cual nos da la unicidad de los coeficientes, en caso de que existan.
La existencia se deduce definiendo

\begin{equation*}
\alpha_{i_1,\dots,i_p}:=F\left(b_{i_1}^{(1)},\dots,b_{i_p}^{(p)}\right),
\end{equation*}
de forma que la condición \eqref{eq:evalbasetensor} se mantiene para
todas las tuplas del tipo \(b_{j_1}^{(1)},\dots,b_{j_p}^{(p)}\). Así, se
tiene la descomposición que buscamos y \(B\) es una base.

\proofe

\prope

\section{Productos tensoriales}\label{productos-tensoriales}

\defineb
Sean \(V_1,V_2,\dots,V_p\) espacios vectoriales. El producto tensorial
de los espacios es el conjunto de funcionales multilineales
\(\LL(V_1^*,V_2^*,\dots,V_p^*;\KK)\), y lo notamos
\(V_1\otimes V_2\otimes\dots\otimes V_p\). \definee

\corb\label{cor:base-tensor}
Sean \(V_1,\dots,V_p\) \(\KK\)-espacios vectoriales con bases
\(B^{(1)},\dots,B^{(p)}\) respectivamente. Llamamos \(b_i^{(k)}\) al
\(i\)-ésimo elemento de la base \(B^{(k)}\) y observamos que podemos
definir el producto tensorial de elementos de \(V_1,\dots,V_p\)
viéndolos como funcionales de \(V_1^*,\dots,V_p^*\). Entonces, la
familia
\[B=\left\{b_{i_1}^{(1)}\otimes\dots\otimes b_{i_p}^{(p)},\ 1\leq i_k\leq\dim{V_k},\ k\in\{1,\dots,p\}\right\},\]
es una base del espacio \(V_1\otimes V_2\otimes\dots\otimes V_p\).
\proofb
Consecuencia de la \autoref{prop:base-funcionales} y el
isomorfismo \(V_k^{**}\cong V_k\). \proofe
\core

\remb
Dados \(v_1\in V_1,\dots v_p\in V_p\), para
\(v'_k\in V_k,\ k\in\{1,\dots,p\},\lambda,\mu\in\KK\) y cualesquiera
\(f_1\in V_1^*,\dots f_p\in V_p^*\) se tiene:

\begin{align*}
  (v_1\otimes v_2\otimes\dots\otimes(\lambda v_k &+ \mu v'_k)\otimes\dots\otimes v_p)(f_1,\dots,f_p)=\\
  f_1(v_1)\dots f_k(\lambda v_k &+ \mu v'_k) \dots f_p(v_p) =\\
  f_1(v_1)\dots (\lambda f_k(v_k) &+ \mu f_k(v'_k)) \dots f_p(v_p) = \\
  \lambda f_1(v_1)\dots f_k(v_k) \dots f_p(v_p) &+ \mu \lambda f_1(v_1)\dots f_k(v'_k) \dots f_p(v_p) =\\
  (\lambda v_1\otimes v_2\otimes\dots\otimes v_k\otimes\dots\otimes v_p &+\mu v_1\otimes v_2\otimes\dots\otimes v'_k\otimes\dots\otimes)(f_1,\dots,f_p)
\end{align*}

Hemos comprobado que la aplicación
\((v_1,v_2,\dots,v_p)\mapsto v_1\otimes v_2\otimes\dots\otimes v_p\) es
lineal en cada variable. \reme

\section{Tensores covariantes y
contravariantes}\label{tensores-covariantes-y-contravariantes}

Sean \(X_1,X_2,\dots X_p,V\) espacios vectoriales y sea \(V_k\) bien
\(X_k\) o bien \(X_k^*\), para cada \(k=1,2,\dots,p\).

\defineb
Decimos que  \(F\in \LL(V_1,V_2,\dots,V_p;V)\)
es una aplicación multilineal \emph{covariante} en la \(k\)-ésima variable si \(V_k=X_k\) y
\emph{contravariante} en dicha variable si \(V_k=X_k^*\).

Si \(F\) es covariante (resp. contravariante) en todas las variables
decimos simplemente que es covariante (resp. contravariante). Si \(F\)
es covariante en \(r\) variables y contravariante en \(s\) variables,
decimos que es \(r\)-covariante \(s\)-contravariante, o de \emph{tipo}
\((r, s)\).\definee

\exampleb Algunos casos particulares de tensores, para observar que generalizan los objetos del álgebra lineal:
\begin{itemize}
\item Dado un espacio vectorial \(V\), un funcional \(f\in V^*\) es un
  tensor 1-covariante.
  \item Un vector \(v\in V\), visto en el doble dual
    \(V^{**}\), es un tensor 1-contravariante.
    \item Por convención, se dice que una constante $\lambda\in\KK$ es un tensor de tipo \((0,0)\).
\end{itemize}
\examplee

\section{Los tensores en aprendizaje
automático}\label{los-tensores-en-aprendizaje-automuxe1tico}

Se ha visto que los tensores generalizan estructuras como los vectores y las aplicaciones lineales. Es importante notar que dichos objetos se pueden representar mediante secuencias finitas de escalares, las componentes, que dependen de la base escogida para los espacios vectoriales donde se esté trabajando. De igual forma, un tensor se puede expresar en componentes del cuerpo $\KK$ respecto de una base.

La expresión de un tensor en componentes necesitará una representación en tantas dimensiones como su rango. Intuitivamente, podemos decir que el rango de un tensor es el número de índices necesarios para recorrer sus componentes.

Esta idea se lleva al aprendizaje automático como una generalización de las matrices. Esencialmente, se le llama \emph{tensor} de rango $p$ a un objeto $T\in \KK^{d_1d_2 \dots d_p}$. Aunque este es un caso particular de los tensores estudiados en este capítulo, no se suele hacer uso de sus propiedades algebraicas. Sin embargo, algunas de las operaciones y descomposiciones de cálculo con tensores permiten realizar un uso eficiente de la memoria disponible en la máquina a la hora de entrenar una red neuronal \autocite{kolda2009}.


\part{Informática}

\chapter{Aprendizaje automático}

En este capítulo introducimos conceptos acerca del aprendizaje automático, la rama de la inteligencia artificial que se ocupa de inferir conocimiento sin necesidad de una programación explícita. Además, se estudia el problema de clasificación y cómo es afectado por la dimensionalidad de los datos. Llegamos así a presentar el problema principal que trata este trabajo: la reducción de la dimensionalidad. Por último, se expone un algoritmo clásico de optimización, gradiente descendente, que servirá como base para los utilizados en Deep Learning.

\section{Introducción}\label{introducciuxf3n}


%\textcolor{red}{qué aborda este capítulo y qué estructura: visión general}

%El aprendizaje profundo o Deep Learning comprende una clase de técnicas
%englobadas dentro del aprendizaje automático. Esta sección introduce los
%conceptos básicos de aprendizaje automático, que son comunes a todas las
%técnicas desarrolladas en el ámbito.

Un \emph{algoritmo de aprendizaje} es, según \textcite{mitchell1997}, un
programa cuyo rendimiento respecto de un conjunto de tareas \(T\) y una
medida de rendimiento \(P\) mejora tras conocer una experiencia \(E\).
En ese caso, se dice que el algoritmo ha \emph{aprendido} de dicha
experiencia.

Estas tareas y experiencias pueden ser de muy diversas clases, lo que
propicia la aparición de algoritmos y técnicas de aprendizaje diferentes
que tratan de abordarlas. Estos algoritmos computacionales son
necesarios cuando la complejidad o el tamaño de la tarea impide tratarla
con técnicas manuales.

Entre las tareas de aprendizaje que se presentan en la literatura se
incluyen:

\begin{itemize}
\tightlist
\item
  clasificación
\item
  regresión
\item
  detección de anomalías
\item
  agrupamiento (\emph{clustering})
\item
  reducción de dimensionalidad
\item
  detección y eliminación de ruido
%\item traducción automática
\end{itemize}


\section{Tipos de aprendizaje}\label{sec:learning-types}

La mayoría de \emph{experiencias} de las que puede
aprender un algoritmo permite categorizar el aprendizaje en dos grandes clases:
supervisado y no supervisado.

% \textcolor{red}{reestructurar}

\subsection{Aprendizaje supervisado}

En aprendizaje supervisado, se le proporciona al algoritmo un conjunto
de ejemplos para los cuales la tarea está resuelta. Así, se pretende que
aprenda a realizar la misma tarea para nuevas muestras.

Algunos problemas concretos en los que se realiza esta clase de aprendizaje son:
\begin{itemize}
\item Clasificación: el programa debe deducir una etiqueta o clase para cada instancia, y para ello el aprendizaje se suele realizar mediante un conjunto de ejemplos que ya tienen asignada su etiqueta. En la \autoref{sec:clasif} se desarrolla este problema en mayor detalle.
\item Regresión: consiste en asignar un valor real a cada nuevo conjunto de valores de entrada, habiendo aprendido de muestras que ya tenían un valor asociado.
\item Predicción de series temporales: se trata de predecir el valor de una variable en el futuro, conociendo su valor en distintos puntos del pasado.
\end{itemize}

\subsection{Aprendizaje no supervisado}\label{aprendizaje-no-supervisado}

La modalidad no supervisada involucra a tareas de las que el algoritmo
no tiene ejemplos resueltos. La experiencia que se le
proporciona puede estar basada en otras características de los datos.

Algunas de las tareas donde se utiliza aprendizaje no supervisado son:
\begin{itemize}
\item Agrupamiento o
\emph{clustering}: se proporcionan al algoritmo datos sin
clasificar que debe subdividir en diferentes conjuntos de forma que los
datos del mismo conjunto sean más similares entre sí que entre elementos
de distintos conjuntos.
\item Reglas de asociación: el programa debe extraer relaciones relevantes entre los
  atributos del conjunto de datos que aporten información útil y novedosa acerca de la
  situación estudiada.
\item Aprendizaje de características: se aportan datos sin etiquetar, para los que el algoritmo debe extraer características representativas. También se da este problema en el aprendizaje supervisado, cuando los datos están etiquetados.
\end{itemize}

El aprendizaje no supervisado abarca multitud de problemas ampliamente
estudiados que tienen diversas aplicaciones presentes en distintos
campos, como el tratamiento de imágenes y reconocimiento de objetos
\autocite{ranzato}, análisis semántico \autocite{hofmann} y sintáctico
del lenguaje \autocite{brent} o el preprocesamiento de datos y
pre-entrenamiento para una posterior fase de aprendizaje
\autocite{erhan2009}.

\section{Problema de clasificación}\label{sec:clasif}

Un problema clásico en el aprendizaje automático es el de clasificación. Se trata de una tarea de aprendizaje supervisado que consiste en aprender acerca de la etiqueta o clase de una secuencia de muestras clasificadas, para después ser capaz de predecir el valor de dicha etiqueta en nuevas instancias sin clasificar.

La clasificación tiene multitud de aplicaciones en diversos ámbitos.
Algunos ejemplos son el diagnóstico de enfermedades \autocite{kononenko2001}, la
detección de fraude \autocite{phua2010} y la clasificación de mensajes de correo \autocite{cohen1996}.

\subsection{Definición}

Una formulación sencilla del problema es la siguiente:

\defineb
Sean \(A_1, A_2, \dots A_f\) conjuntos no vacíos llamados
\emph{atributos de entrada}. Llamaremos \emph{espacio de atributos} (o
\emph{espacio de características}) a
\(\mathcal A=A_1\times A_2\times\dots\times A_f\).

Sea $L$ un conjunto finito al que denominaremos \emph{conjunto de etiquetas}.

Sea \(D\subset \mathcal A\times L\) un subconjunto finito del espacio de
atributos, lo llamaremos \emph{conjunto de instancias} o \emph{dataset}.

Decimos que la tripleta \(\mathcal P=\left(\mathcal A, L, D\right)\) es
un \emph{problema de clasificación}. \definee

\defineb
Dado un problema de clasificación \(\left(\mathcal A, L, D\right)\), un
\emph{clasificador} es una aplicación \(c:\mathcal A\rightarrow L\).
\definee

Así, el objetivo que se persigue al abordar un problema de clasificación
\(\mathcal P\) es encontrar el clasificador \(c\) que mejor se adapte al
problema, según una o varias métricas de evaluación. Intuitivamente, el
procedimiento por el que se obtenga dicho clasificador debe ser capaz de
utilizar la información de las instancias en el dataset \(D\) para
predecir una clase en nuevas instancias del espacio de atributos.

Atendiendo a la estructura de $L$, es decir, el número de características que estarán ausentes en los nuevos datos y sus posibles valores, distinguiremos los siguientes tipos de clasificación:
\begin{itemize}
\item Binaria: implica clasificar en 2 clases (generalmente significan que una condición es verdadera o falsa), utilizando una característica que tome únicamente dos valores. Así, $L = \{0,1\}$.
\item Multiclase: en este caso habrá más de dos clases, pero cada instancia pertenecerá a una y solo una de ellas, por lo que se usará una característica que contenga tantos valores como clases: $L=\{0, 1,\dots, l\}$.
\item Multietiqueta: en esta situación, cada instancia puede asociarse a más de una etiqueta, por tanto se usarán tantos atributos como etiquetas, cada uno de ellos conteniendo dos valores: $L=\{0,1\}\times\dots\times\{0,1\}$.
\item Multidimensional: se trata de una generalización del caso multietiqueta donde cada una de las etiquetas toma valores en un conjunto finito arbitrario, esto es, $L=\{0,\dots, \lambda_1\}\times\dots\times\{0,\dots, \lambda_l\}$.
\end{itemize}


Las definiciones previas componen una formalización simple del problema
de clasificación. Una modelización más detallada y con consecuencias
teóricas de interés se encuentra en la teoría de aprendizaje PAC
\autocite{shwartz2014}. De esta teoría además se extraen algunos resultados
relevantes. Por un lado, el hecho de que los algoritmos sean capaces de generalizar
un modelo adecuado a partir de una cantidad finita de muestras. Por contraposición,
considerados sobre el conjunto de todas las distribuciones de datos
posibles, todos los algoritmos de clasificación presentan en media la
misma tasa de error en la predicción de clases para nuevos ejemplos
(hecho conocido como el teorema de \textit{No Free Lunch}\textcolor{red}{ref!}).

%\begin{example}
%\textcolor{red}{Si se me ocurre un ejemplo, añadirlo.}
%\end{example}

\subsection{Estructura del espacio de
atributos}\label{estructura-del-espacio-de-atributos}

En principio no tenemos por qué asumir una estructura algebraica para el
espacio de atributos \(\mathcal A\), pero la mayoría de algoritmos
necesitarán una forma de medir similitud entre instancias. Para ello,
normalmente se puede utilizar una distancia \(d\), de forma que
\((\mathcal A,d)\) sea un espacio métrico. Para el uso de un conjunto de
datos en redes neuronales, sin embargo, convendrá suponer además
\(\mathcal A\subset \mathbb R^f\).

\section{Problema de reducción de
dimensionalidad}\label{sec:red-dim}

Es común encontrar problemas de clasificación donde el espacio de
atributos posee una alta dimensionalidad. Por ejemplo: conjuntos de
datos extraídos de texto, donde cada atributo representa la aparición de
una palabra en un documento; conjuntos basados en imágenes donde cada
atributo representa un píxel \autocite{mnist}, o datos que
expresan características genéticas \autocite{clarke2008}.

Como consecuencia del \autoref{th:dim-curse}, al aumentar la dimensionalidad de los conjuntos de datos se pierde significatividad en las distancias, en el sentido de que, para una instancia de la muestra dada, la instancia más cercana y la instancia más lejana están a distancias muy similares.

Las distancias entre puntos en el espacio de atributos son utilizadas en
multitud de algoritmos de aprendizaje automático, entre los cuales el
ejemplo más claro es la técnica del Vecino más cercano
\autocite{peterson2009}, aplicada en clasificación en el algoritmo de los k-vecinos más cercanos.

Ante un conjunto de datos de alta
dimensionalidad, podemos discernir dos vías de acción:
\begin{itemize}
\tightlist
\item
  Estudiar y transformar los datos manteniendo la dimensionalidad, para
  que las distancias entre puntos sean significativas.
\item
  Reducir la dimensionalidad de los datos, manteniendo toda la
  información útil que sea posible.
\end{itemize}
En nuestro caso, utilizaremos algunas técnicas de Deep Learning para
operar de la segunda forma, comprimiendo los datos en un espacio de
menor dimensionalidad.

Con el objetivo de reducir la dimensionalidad de un conjunto de datos,
se alteran las características del mismo en un proceso denominado
\emph{extracción de características}, que comprende dos procesos alternativos que en ocasiones se combinan
\autocite{guyon2006}:

\begin{enumerate}
\def\labelenumi{\arabic{enumi}.}
\tightlist
\item
  \textbf{Construcción de características}. Se transforman los datos,
  operando entre las características para hallar un nuevo conjunto de
  atributos que posiblemente facilite el aprendizaje de los datos.
\item
  \textbf{Selección de características}. Se escogen las características
  que se consideren más relevantes para obtener información, y se
  descartan las que no son útiles.
\end{enumerate}

En ocasiones únicamente es necesaria una selección de características y
se puede obviar la primera fase. En otras, la propia construcción de
características sirve para sustituir a las originales y por tanto
incluye la segunda etapa.

La selección de características se puede realizar mediante muy distintas
técnicas, desde metaheurísticas hasta basadas en teoría de la
información \autocite{molina2002}.

Por otro lado, la construcción de características se pone en práctica de
formas de variable complejidad. De entre las más sencillas cabe
mencionar la normalización, la discretización, o tomar combinaciones
lineales de las características existentes. Además, se pueden destacar
técnicas más avanzadas como Isomap \autocite{tenenbaum2000}, Locally
Linear Embedding \autocite{roweis2000}, o los autoencoders
\autocite{hinton2006autoencoder}. Este trabajo se centrará en estudiar
estos últimos.

\section{Métodos de optimización: gradiente
descendente}

\label{sec:grad-desc}

Gran parte del trabajo computacional en aprendizaje automático requiere
optimizar funciones, es decir, encontrar el máximo o el mínimo de una
función \(f:\RR^n\rightarrow \RR\). Dicha función se suele denominar
\emph{función objetivo}, y normalmente corresponde al coste o error de
aprendizaje de un algoritmo. En lo sucesivo, se supondrá que el objetivo
a conseguir es encontrar el mínimo de la función \(f\). Los resultados y
algoritmos son análogos para la búsqueda de un máximo, simplemente
cambiando el signo de \(f\).

Estudiemos una simplificación del problema de optimización a una
variable. Sea \(f:[a,b]\rightarrow\RR\) continua en \([a,b]\) y
derivable con derivada continua en \(]a,b[\). Por un resultado elemental
de análisis matemático, es conocido que un extremo (máximo o mínimo)
local en una función derivable se encuentra siempre en un punto de
derivada cero. Además, por el teorema del valor medio, para
\(\varepsilon>0\) se tiene que
\[\exists c\in ]a, a+\varepsilon[ :\ f(a+\varepsilon)-f(a)=\varepsilon f'(c)\Rightarrow f(a+\varepsilon)=f(a) + \varepsilon f'(c)\]
y, por continuidad de \(f'\), si \(\varepsilon\) es suficientemente
pequeño el signo de la derivada no cambiará entre \(a\) y \(c\). Así,
\[f'(a)<0\Rightarrow f(a+\varepsilon) < f(a);\ f'(a)>0\Rightarrow f(a+\varepsilon) > f(a).\]

El método del gradiente descendente, en una variable, consiste en
evaluar \(f\) a pequeños saltos de \(\varepsilon\) y consultar la
derivada para decidir el sentido del próximo salto.

En el caso de varias variables, el algoritmo es muy similar. Sabiendo
que el gradiente de una función real de varias variables
\(f:\RR^n\rightarrow \RR\) es un vector que apunta en la dirección y
sentido de la mayor pendiente (derivada direccional) ascendente de la
función, el gradiente cambiado de signo estará posicionado en el sentido
de mayor pendiente descendente.

Así, suponiendo la suficiente regularidad para \(f\) y que se puede
evaluar tanto la función como su derivada en cualquier punto del
dominio, se puede aplicar el algoritmo de gradiente descendente,
originalmente debido a \textcite{cauchy1847}.

El algoritmo generalizado consiste en, a cada paso determinado por un
punto \(x\) del dominio, consultar el gradiente de \(f\) en \(x\),
\(\nabla f(x)\), y ``saltar'' una cantidad \(\varepsilon>0\) en su
dirección y en el sentido contrario: \[x' = x - \varepsilon\nabla f(x)\]
Al escalar \(\varepsilon\) se le suele llamar \emph{tasa de
aprendizaje}.

El algoritmo termina cuando \(\nabla f(x)\) es el vector cero o muy
cercano a cero, con una tolerancia dada.

\begin{figure}[hbtp]
  \centering
  \includegraphics[width=0.45\textwidth]{images/gradient_ascent_contour.png}
  \includegraphics[width=0.45\textwidth]{images/gradient_ascent_surface.png}
  \caption{\label{fig:grad-desc}Visualización de la técnica de gradiente descendente, en este caso, buscando un máximo de la función $F(x,y)=\sin\left(\frac{1}{2} x^2 - \frac{1}{4} y^2 + 3 \right) \cos(2 x+1-e^y)$. Imágenes de Wikimedia Commons en dominio público}
\end{figure}

La técnica de gradiente descendente presenta algunos problemas: como se
puede observar en la \autoref{fig:grad-desc}, cuando el gradiente de
la función es próximo a cero el algoritmo tiende a dar pasos muy cortos,
convergiendo muy lentamente hacia el extremo local encontrado. Asimismo,
en general puede presentar un comportamiento de zigzag para ciertas
funciones, avanzando de forma casi ortogonal al segmento que guarda la
distancia más corta con el extremo.


\chapter{Deep Learning}
\section{Redes neuronales prealimentadas
profundas}\label{redes-neuronales-prealimentadas-profundas}

\label{sec:feedforward}

Las redes prealimentadas profundas, también conocidas como perceptrones
multicapa o en inglés como \emph{deep feedforward neural networks}, son
el modelo canónico de aprendizaje profundo \autocite{goodfellow2016}. El
objetivo de una red prealimentada es aproximar una función \(f^{*}\),
definiendo una aplicación \(f(x;\theta)\) y aprendiendo el valor de los
parámetros \(\theta\) que resultan en la mejor aproximación.

En concreto, las redes prealimentadas se caracterizan por que no se
forman ciclos en las conexiones entre unidades. Así, la información se
evalúa siempre hacia adelante a través de las conexiones intermedias
usadas para definir \(f\), hasta la salida de la red. No hay
retroalimentaciones en las que salidas de algunas unidades de la red
vuelvan a ser entradas del modelo.

Estas redes se suelen representar como una composición en cadena de
varias funciones, que se puede asociar a un grafo acíclico. Por ejemplo,
podríamos tener una red composición de funciones vectoriales
\(f_1, f_2, f_3\) de la siguiente forma: \(f(x)=f_3(f_2(f_1(x)))\). En
este caso, decimos que \(f_1\) es la primera capa, \(f_2\) la segunda
capa y \(f_3\) la capa de salida. Las capas que no corresponden a la
salida de \(f\) se suelen denominar \emph{capas ocultas}. La longitud de
esta cadena nos da la profundidad del modelo.

A diferencia de otros algoritmos de aprendizaje automático, las redes
neuronales mantienen esta estructura de capas de forma que la capa
\(i+1\)-ésima únicamente opera con los datos de salida de la
\(i\)-ésima; en particular, sólo la primera capa utiliza directamente
los datos de entrada. Además, por la inspiración biológica de las redes,
cada componente de cada capa (\emph{unidad}) se puede interpretar como
una neurona, actuando como una función de \(\RR^{n_{i-1}}\) en \(\RR\),
donde \(n_{i-1}\) es el número de componentes de la capa anterior. El
comportamiento es similar a una neurona en el sentido de que recoge
información de varias unidades cercanas y calcula su propio valor de
activación, así como la estructuración en capas se ha tomado de la
neurociencia. En la figura \ref{fig:dfnn} se muestra una representación
común de una red neuronal como unidades conectadas formando un grafo.
Las flechas indican el sentido en el que viajan los datos, es decir, las
salidas de funciones que se toman como entradas de otras funciones.

\begin{figure}[hbtp]
  \centering
\begin{tikzpicture}[scale=0.2]
\tikzstyle{every node}+=[inner sep=0pt]
\draw [black] (21.5,-13.4) circle (3);
\draw [black] (21.5,-23.4) circle (3);
\draw [black] (21.5,-33.1) circle (3);
\draw [black] (36.4,-7.6) circle (3);
\draw [black] (36.4,-17.8) circle (3);
\draw [black] (36.4,-28.2) circle (3);
\draw [black] (36.4,-39.6) circle (3);
\draw [black] (50.6,-23.4) circle (3);
\draw [black] (24.35,-32.16) -- (33.55,-29.14);
\fill [black] (33.55,-29.14) -- (32.63,-28.91) -- (32.95,-29.86);
\draw [black] (24.25,-34.3) -- (33.65,-38.4);
\fill [black] (33.65,-38.4) -- (33.12,-37.62) -- (32.72,-38.54);
\draw [black] (23.59,-30.95) -- (34.31,-19.95);
\fill [black] (34.31,-19.95) -- (33.39,-20.17) -- (34.11,-20.87);
\draw [black] (23.01,-30.51) -- (34.89,-10.19);
\fill [black] (34.89,-10.19) -- (34.05,-10.63) -- (34.91,-11.13);
\draw [black] (23.56,-21.22) -- (34.34,-9.78);
\fill [black] (34.34,-9.78) -- (33.43,-10.02) -- (34.16,-10.71);
\draw [black] (24.31,-22.34) -- (33.59,-18.86);
\fill [black] (33.59,-18.86) -- (32.67,-18.67) -- (33.02,-19.6);
\draw [black] (24.36,-24.32) -- (33.54,-27.28);
\fill [black] (33.54,-27.28) -- (32.94,-26.56) -- (32.63,-27.51);
\draw [black] (23.53,-25.61) -- (34.37,-37.39);
\fill [black] (34.37,-37.39) -- (34.2,-36.46) -- (33.46,-37.14);
\draw [black] (24.3,-12.31) -- (33.6,-8.69);
\fill [black] (33.6,-8.69) -- (32.68,-8.51) -- (33.04,-9.44);
\draw [black] (24.38,-14.25) -- (33.52,-16.95);
\fill [black] (33.52,-16.95) -- (32.9,-16.24) -- (32.61,-17.2);
\draw [black] (23.63,-15.51) -- (34.27,-26.09);
\fill [black] (34.27,-26.09) -- (34.06,-25.17) -- (33.35,-25.88);
\draw [black] (22.98,-16.01) -- (34.92,-36.99);
\fill [black] (34.92,-36.99) -- (34.96,-36.05) -- (34.09,-36.54);
\draw [black] (38.41,-9.83) -- (48.59,-21.17);
\fill [black] (48.59,-21.17) -- (48.43,-20.24) -- (47.69,-20.91);
\draw [black] (39.19,-18.9) -- (47.81,-22.3);
\fill [black] (47.81,-22.3) -- (47.25,-21.54) -- (46.88,-22.47);
\draw [black] (39.24,-27.24) -- (47.76,-24.36);
\fill [black] (47.76,-24.36) -- (46.84,-24.14) -- (47.16,-25.09);
\draw [black] (38.38,-37.34) -- (48.62,-25.66);
\fill [black] (48.62,-25.66) -- (47.72,-25.93) -- (48.47,-26.59);
\end{tikzpicture}
\caption{\label{fig:dfnn} Ilustración ejemplificando una red neuronal prealimentada de tres capas}
\end{figure}

Para entender cómo las redes prealimentadas aproximan funciones,
consideremos algunos modelos lineales como la regresión lineal o la
logística. Estos modelos tienen claras ventajas, son sencillos, se
pueden ajustar de forma eficiente y fiable. Sin embargo, están muy
limitados, dado que sólo tiene sentido aplicarlos a funciones lineales,
por lo que no pueden sintetizar interacciones entre dos variables de
entrada.

Cuando el objetivo es aproximar funciones no lineales, una vía es
aplicar un modelo lineal no a la variable independiente sino a una
transformación no lineal de la misma. El problema se traduce entonces en
qué transformación \(\phi\) de la variable aplicar para que el modelo
lineal tenga un buen ajuste. Frente a buscar \(\phi\) manualmente, que
requiere extenso conocimiento de cada problema, o usar un \(\phi\) de
muy alta dimensionalidad con capacidad para todos los ejemplos del
conjunto de datos, las redes neuronales realizan un aprendizaje de
\(\phi\) entre una clase de funciones parametrizada: se define un modelo
del tipo

\begin{equation}
  f^{*}(x)\approx f(x;\theta,w)=\Tr{\phi(x;\theta)}w, 
\end{equation}

donde \(\theta\) es un vector de parámetros que facilita escoger una
función \(f\) concreta de entre la clase que define, y \(w\) es otro
vector de parámetros que permite aplicar la transformación obtenida en
la salida deseada. Para encontrar los parámetros que corresponden a una
buena aproximación, se utilizará un algoritmo de optimización basado en
la técnica de gradiente descendiente vista en la sección
\ref{sec:grad-desc}. Se trata de un enfoque muy flexible, ya que se
puede proveer al algoritmo de una clase de funciones más general o más
concreta, según el conocimiento sobre el problema que se posea.

La clase de funciones, dentro de la cual una red neuronal busca la
aproximación, se determina escogiendo la estructura de la red y los
tipos de unidades ocultas y de salida.

\subsection{Funciones de coste}\label{funciones-de-coste}

La mayoría de diseños de redes neuronales involucran definir una
distribución \(P(y\mid x;\theta)\) y aplicar el principio de máxima
verosimilitud. En otros casos, mediante funciones de coste específicas,
se puede predecir simplemente algún estadístico de \(y\) condicionado a
\(x\), en lugar de determinar una distribución de probabilidad.

En el caso más habitual, la función de coste se definirá como la
entropía cruzada (equivalentemente, la log-verosimilitud negativa) entre
la distribución de los datos, \(\hat p\), y la del modelo, \(p\), y
sobre la variable de la salida generada \(y\) respecto de la entrada
\(x\):
\[J(\theta)=C(\hat p(y\mid x), p(y\mid x))=-\E_{\hat p}[\log p(y\mid x)].\]
Una ventaja de este enfoque que esta función de coste viene determinada
automáticamente por el modelo \(p(y\mid x)\) que escojamos y evita tener
que definir una nueva función para cada modelo. Además, la entropía
cruzada suele permitir calcular gradientes relativamente ``grandes'', en
el sentido de que no se acercan rápidamente a cero, lo cual beneficia al
proceso de optimización.

En ocasiones se añade a la función de coste un término de regularización
o \emph{decaimiento de pesos} de forma que el coste total queda:
\[J(\theta)=C(\hat p, p;y\mid x) + \lambda \Omega(\theta)\]

\subsection{Unidades de salida}\label{unidades-de-salida}

La expresión concreta de la función de coste, cuando la tomamos como la
entropía cruzada, vendrá determinada por la representación de la salida
de la red prealimentada. Estudiamos a continuación el tipo de unidades
que se suelen utilizar para dar dicha salida. Durante el resto de esta
sección, supondremos que la red proporciona un vector de características
\(h=f(x;\theta)\) generado por las unidades ocultas. El cometido de las
unidades de salida es dar una transformación que aporte una salida
apropiada.

\subsubsection{Unidades lineales}\label{unidades-lineales}

Una capa de unidades de este tipo realiza una transformación afín de los
datos: \(\hat y=\Tr Wh+b\). Se suelen utilizar para calcular la media de
una distribución condicional normal: \[p(y\mid x)=\PN(y;\hat t,I).\] En
ese caso, maximizar la entropía cruzada es equivalente a minimizar el
error cuadrático medio.

\subsubsection{Unidades con activación
sigmoidal}\label{unidades-con-activaciuxf3n-sigmoidal}

En muchas tareas, la variable objetivo \(y\) es de tipo binario. Por
ejemplo, los problemas de clasificación binaria son un caso particular
de esta situación. La técnica de máxima verosimilitud lleva a definir
una distribución de Bernoulli sobre \(y\) condicionada a \(x\). Esta
distribución está determinada por un único número en el intervalo
\([0, 1]\), que se corresponde con \(P(y=1\mid x)\).

Para que la unidad de salida tenga un buen comportamiento, es necesario
que no genere gradiente 0 ni muy cercano a 0 en casos en los que el
modelo no se acerque a una solución. Esto se consigue utilizando una
\emph{función de activación}, es decir, se compone el cómputo de la
unidad con otra función que la regulariza de alguna manera. En este
caso, se utiliza la función logística:
\[z=\Tr wh+b;\ \hat y = \sigma(z) = \frac{1}{1+e^{-z}}\]

Definamos ahora una distribución de probabilidad sobre \(y\), usando el
valor \(z\). Podemos comenzar asumiendo que la probabilidad no
normalizada \(\tilde P\) es log-lineal en \(z\) e \(y\):
\[\log \tilde P(y)=yz,\] y exponenciamos \[\tilde P(y)=e^{yz},\] para
ahora normalizar sobre los valores de \(\tilde P\), obteniendo una
probabilidad \[P(y)=\frac{e^{yz}}{e^{0z}+e^{1z}}=\frac{e^{yz}}{1+e^z}\]
y utilizando de nuevo que \(y\in\{0,1\}\) se tiene

\begin{equation}\label{eq:sigm-prob}
  P(y)=\frac{e^{yz}}{e^{(1-y)z}+e^{yz}}=\frac{1}{\frac{e^{z}}{e^{2yz}}+1}=\sigma((2y-1)z).
\end{equation}

El resultado es una distribución de Bernoulli determinada por una
transformación logística de \(z\). De hecho, puesto que los posibles
valores de \(y\) son 0 y 1, también podemos expresarla más claramente
como:
\[P(y)=\sigma(z)^y\sigma(-z)^{1-y}=p^y(1-p)^{1-y}\mbox{ donde }p=\sigma(z).\]

Ahora, la función de coste para esta distribución, tomando la entropía
cruzada y usando \eqref{eq:sigm-prob}, es:
\[J(\theta)=-\log P(y\mid x)=-\log\sigma((2y-1)z).\]

Dado que la función logística está valuada en el intervalo abierto
\(]0,1[\), su logaritmo es finito y \(J\) está bien definida.

\subsubsection{\texorpdfstring{Unidades con activación
\emph{softmax}}{Unidades con activación softmax}}\label{unidades-con-activaciuxf3n-softmax}

Las unidades con función de activación \emph{softmax} se emplean cuando
se pretende representar una distribución de probabilidad sobre una
variable discreta con un número finito de valores. Generalmente, esta
situación se da en problemas de clasificación multiclase. Así, se pueden
interpretar como una generalización de las unidades sigmoidales.

Mientras que para caracterizar una variable binaria bastaba con una sola
unidad de salida (la salida era un escalar entre 0 y 1), ahora se
utilizarán tantas unidades como posibles valores de la variable. Si
\(y\) puede tomar uno de entre \(n\) posibles valores, la capa de salida
con \emph{softmax} generará un vector \(\hat y\) donde
\(\hat y_i=P(y=i\mid x)\), exigiendo que \(\sum_{i=1}^n\hat y_i=1\).

La función vectorial \emph{softmax} se define en cada componente
\(i=1,\dots,n\) como

\begin{equation}\label{eq:softmax}
  \softmax{z}_i = \frac{\exp(z_i)}{\sum_{j=1}^{n}\exp(z_j)}.
\end{equation}

El vector \(z\) al que se aplica la función se obtiene de una capa de
unidades lineales que proporcionan probabilidades logarítmicas sin
normalizar: \[z=\Tr Wh+b;\ z_{i}=\log\tilde P(y=i\mid x).\] De nuevo, la
función de coste se puede definir siguiendo la misma técnica, mediante
la entropía cruzada.

\subsection{Unidades ocultas}\label{unidades-ocultas}

Cualquiera de los tipos anteriores de unidad se puede utilizar en una
capa oculta, pero existen más y el diseño de unidades ocultas es un
campo de investigación muy activo, pese a la falta de principios
teóricos que lo guíen. A continuación se exponen algunos de los tipos
más usuales. Salvo que se indique lo contrario, todas las capas de
unidades calculan una transformación afín \[z=\Tr Wx+b,\] donde \(x\) es
el vector de entrada, equivalentemente, el vector de salida de la capa
inmediatamente anterior o un vector de datos si se trata de la primera
capa. Se compone esta transformación con la función de activación
específica a cada tipo de unidad.

\subsubsection{Unidades lineales rectificadas
(ReLU)}\label{unidades-lineales-rectificadas-relu}

La función de activación de las unidades lineales rectificadas
(\emph{Rectified Linear Units}, ReLU) es

\begin{equation}
g(z)_i=\max\{0,z_i\}.
\end{equation}

Estas unidades son fáciles de optimizar ya que son similares a las
unidades lineales. Aunque no es diferenciable en \(z=0\), se pueden
utilizar algoritmos basados en gradiente para optimizar la función
objetivo. Esto es debido a que, generalmente, durante el entrenamiento
no se llega a un punto en el que el gradiente sea exactamente 0, así que
se pueden aceptar puntos donde el gradiente no esté definido en los
mínimos de la función coste.

\subsubsection{Extensiones de ReLU}\label{extensiones-de-relu}

Algunas generalizaciones de las ReLU modifican el gradiente cuando las
componentes de \(z\) son negativas:

\begin{itemize}
\tightlist
\item
  \textbf{Rectificación por valor absoluto}:
  \(g(z)_{i}=\lvert z_{i} \rvert\)
\item
  \textbf{\emph{Leaky} ReLU}: \(g(z)_i=\max(0,z_i)+\alpha\min(0,z_i)\)
  con \(\alpha\) pequeño como \(0,01\)
\item
  \textbf{ReLU paramétrica}: \(g(z)_i=\max(0,z_i)+\alpha_i\min(0,z_i)\),
  con \(\alpha_i\) como un parámetro optimizable
\end{itemize}

Las capas de \textbf{unidades \emph{maxout}} agrupan las componentes de
\(z\) en conjuntos de \(k\) valores cada uno. Cada una de las unidades
de la capa proporciona entonces el máximo de uno de esos conjuntos:
\[g(z)_i=\max\{z_j:j\in S_i\},\ S_i=\{(i-1)k + 1, (i-1)k + 2,\dots, ik\}.\]
En este caso, si \(z\in\RR^{kd}\), entonces \(g(z)\in\RR^{d}\). El
aspecto interesante de las unidades \emph{maxout} es el hecho de que una
capa de ellas puede aprender una función convexa lineal a trozos de
hasta \(k\) trozos. Podemos intuir que una capa de este tipo podrá
aproximar cualquier función convexa con precisión arbitraria para un
\(k\) conveniente.

\subsubsection{Unidades con activación sigmoidal o tangente
hiperbólica}\label{unidades-con-activaciuxf3n-sigmoidal-o-tangente-hiperbuxf3lica}

Otras dos funciones muy comunes para la activación de unidades en redes
neuronales son la función logística

\begin{equation}
  g(z)_i=\sigma(z_i)=\frac{1}{1+e^{-z_i}},
\end{equation}

y la tangente hiperbólica

\begin{equation}
  g(z)_i=\tanh(z_i)=\frac{e^{z_i}-e^{-z_i}}{e^{z_i}+e^{-z_i}}.
\end{equation}

Se puede comprobar que \(\tanh(z_i)=2\sigma(2z_i)-1\).

El uso de estas unidades era más común antes de la aparición de las
ReLU. En la actualidad está decreciendo su uso.

\subsubsection{Otras unidades ocultas}\label{otras-unidades-ocultas}

Para construir otros tipos de unidad oculta simplemente basta con elegir
otra función de activación. Las siguientes son algunas relativamente
comunes:

\begin{itemize}
\tightlist
\item
  El \textbf{coseno} \(g(z)_i=cos(z_i)\) ha sido usada por
  \textcite{goodfellow2016} en el conocido conjunto MNIST obteniendo una
  tasa de error inferior al 1\%.
\item
  La \textbf{identidad} \(g(z) = z\) se puede utilizar para encadenar
  capas de forma lineal y utilizar alguna función de activación
  diferente a la salida.
\item
  La función \emph{softplus} \(g(z)_i=\zeta(z_i)=\log(1+e^{z_i})\) es
  una versión infinitamente derivable de la unidad lineal rectificada.
  Sin embargo, en la práctica no suele presentar ventajas sobre la ReLU.
\end{itemize}

\section{Entrenamiento de redes neuronales
profundas}\label{entrenamiento-de-redes-neuronales-profundas}

\subsection{Propagación hacia
adelante}\label{propagaciuxf3n-hacia-adelante}

Las redes neuronales prealimentadas, que se han estudiado en la sección
\ref{sec:feedforward}, son funciones que aceptan vectores de entrada y
procesan la información computando varias funciones intermedias,
propagando así la información, hasta la salida de la red. Este proceso
se denomina \emph{propagación hacia adelante}, y se describe en el
algoritmo \ref{alg:fwdprop}.

\begin{algorithm}
\caption{Propagación hacia adelante en una red neuronal profunda con función de activación $g$, y cálculo de la función de coste $J$, para una instancia $x$ (en la práctica se utilizan minilotes de instancias)}
\label{alg:fwdprop}
\begin{algorithmic}
  \REQUIRE{profundidad de la red $l$}
  \REQUIRE{$W^{(i)}$ matriz de pesos de la capa $i$-ésima}
  \REQUIRE{$b^{(i)}$ vector de sesgos de la capa $i$-ésima}
  \REQUIRE{instancia $x$ a procesar}
  \REQUIRE{salida objetivo $y^{*}$}
  \STATE{$h^{(0)}\gets x$}
  \FOR{$k=1,\dots,l$}
  \STATE{$z^{(k)}\gets W^{(k)}h^{(k-1)}+b^{(k)}$}
  \STATE{$h^{(k)}\gets g(z^{(k)})$}
  \ENDFOR
  \STATE{$y\gets h^{(l)}$}
  \STATE{Se calcula la función de coste mediante una distancia o pérdida entre la salida obtenida y la deseada, y un término de regularización $\Omega$:}
  \STATE{$J\gets L(y,y^{*})+\lambda \Omega(\theta)$}
\end{algorithmic}
\end{algorithm}

\subsection{Propagación hacia
atrás}\label{propagaciuxf3n-hacia-atruxe1s}

Consideremos una red neuronal prealimentada profunda, determinada por un
vector de parámetros \(\theta\). Durante el entrenamiento, la salida
generada por la propagación hacia adelante se compara con la salida
deseada y se calcula un coste \(J(y,y^{*};\theta)\in\RR\). Para aplicar
un algoritmo de optimización basado en gradiente descendiente (se
estudiarán en la sección \ref{sec:dl-opt}), es necesario conocer el
gradiente de la función \(J\) respecto de los parámetros \(\theta\).
Este gradiente se puede calcular analíticamente, pero evaluar
\(\nabla_{\theta} J(y,y^{*};\theta)\) es generalmente muy costoso
computacionalmente. El algoritmo de propagación hacia atrás, o
\emph{backprop}, realiza este cálculo de forma eficiente.

\begin{example}
  
Consideremos la red neuronal de la figura \ref{fig:ex-backprop}. Suponiendo que cada neurona utiliza la misma función de activación $g$, podemos dar una expresión de $f$ acorde con los pesos y sesgos que se muestran en la imagen.

  \begin{figure}[hbtp]
\centering
\begin{tikzpicture}[scale=0.2]
\tikzstyle{every node}+=[inner sep=0pt]
\draw [black] (34,-24.7) circle (3);
\draw [black] (34,-38.8) circle (3);
\draw [black] (48.7,-33) circle (3);
\draw (15.2,-24.7) node {$x_1$};
\draw (15.2,-38.8) node {$x_2$};
\draw (61.1,-33) node {$f$};
\draw (23.8,-13.6) node {$1$};
\draw (41.5,-13.6) node {$1$};
\draw [black] (36.79,-37.7) -- (45.91,-34.1);
\fill [black] (45.91,-34.1) -- (44.98,-33.93) -- (45.35,-34.86);
\draw (43.4,-36.44) node [below] {$w_{12}^{(2)}$};
\draw [black] (36.61,-26.18) -- (46.09,-31.52);
\fill [black] (46.09,-31.52) -- (45.64,-30.7) -- (45.15,-31.57);
\draw (39.21,-29.35) node [below] {$w_{11}^{(2)}$};
\draw [black] (17.6,-26.5) -- (31.6,-37);
\fill [black] (31.6,-37) -- (31.26,-36.12) -- (30.66,-36.92);
\draw (19.45,-29.25) node [below] {$w_{12}^{(1)}$};
\draw [black] (17.6,-37) -- (31.6,-26.5);
\fill [black] (31.6,-26.5) -- (30.66,-26.58) -- (31.26,-27.38);
\draw (22.75,-33.50) node [below] {$w_{21}^{(1)}$};
\draw [black] (51.7,-33) -- (58.1,-33);
\fill [black] (58.1,-33) -- (57.3,-32.5) -- (57.3,-33.5);
\draw [black] (25.83,-15.81) -- (31.97,-22.49);
\fill [black] (31.97,-22.49) -- (31.8,-21.56) -- (31.06,-22.24);
\draw (31.36,-16.61) node [left] {$b_1^{(1)}$};
\draw [black] (42.54,-16.41) -- (47.66,-30.19);
\fill [black] (47.66,-30.19) -- (47.85,-29.26) -- (46.91,-29.61);
\draw (44.34,-24.11) node [left] {$b_1^{(2)}$};
\draw [black] (24.93,-16.38) -- (32.87,-36.02);
\fill [black] (32.87,-36.02) -- (33.04,-35.09) -- (32.11,-35.47);
\draw (25.16,-20.1) node [left] {$b_2^{(1)}$};
\draw [black] (18.2,-24.7) -- (31,-24.7);
\fill [black] (31,-24.7) -- (30.2,-24.2) -- (30.2,-25.2);
\draw (24.6,-25.2) node [below] {$w_{11}^{(1)}$};
\draw [black] (18.2,-38.8) -- (31,-38.8);
\fill [black] (31,-38.8) -- (30.2,-38.3) -- (30.2,-39.3);
\draw (24.6,-39.3) node [below] {$w_{22}^{(1)}$};
\end{tikzpicture}
\caption{\label{fig:ex-backprop}Red neuronal sencilla de dos capas con entrada vectorial de dos componentes, se marcan los pesos y los sesgos en cada conexión}
\end{figure}

En este caso, el vector de parámetros que determina la red será
$$\theta=\left(w_{11}^{(1)},w_{12}^{(1)},w_{21}^{(1)},w_{22}^{(1)},b_{1}^{(1)},b_{2}^{(1)},w_{11}^{(2)},w_{12}^{(2)},b_{1}^{(2)}\right).$$
Puesto que la función de coste $J$ vendrá determinada por el valor de $f$ en el mismo punto, para calcular su gradiente nos interesa conocer el de $f$. La expresión desarrollada de $f$ queda
\begin{gather*}
f(x_1,x_2;\theta)=\\g\left(w_{21}^{(2)}g\left(w_{11}^{(1)}x_{1}+w_{12}^{(1)}x_2+b_1^{(1)}\right)+w_{22}^{(2)}g\left(w_{21}^{(1)}x_1+w_{22}^{(1)}x_2+b_2^{(1)}\right)+b_1^{(2)}\right).
\end{gather*}

Ahora, mediante la regla de la cadena podemos desarrollar la parcial de $f$ respecto de cualquiera de los parámetros. Llamamos
\begin{align*}
  \alpha&=g'\left(w_{21}^{(2)} g\left(w_{11}^{(1)}x_{1}+w_{12}^{(1)}x_2+b_1^{(1)}\right) + w_{22}^{(2)} g\left(w_{21}^{(1)}x_1+w_{22}^{(1)}x_2+b_2^{(1)}\right)+b_1^{(2)}\right)\\
  \beta&=g'\left(w_{11}^{(1)}x_{1}+w_{12}^{(1)}x_2+b_1^{(1)}\right)\\
  \gamma&=g'\left(w_{21}^{(1)}x_{1}+w_{22}^{(1)}x_2+b_2^{(1)}\right)
\end{align*}
y se tiene
\begin{alignat*}{3}
  \frac{\partial f}{\partial w_{11}^{(1)}}(x_1,x_2;\theta)&=\alpha w_{11}^{(2)}\beta x_1,\quad&
  \frac{\partial f}{\partial w_{12}^{(1)}}(x_1,x_2;\theta)&=\alpha w_{11}^{(2)}\beta x_2,\\
  \frac{\partial f}{\partial w_{21}^{(1)}}(x_1,x_2;\theta)&=\alpha w_{12}^{(2)}\gamma x_1,\quad&
  \frac{\partial f}{\partial w_{22}^{(1)}}(x_1,x_2;\theta)&=\alpha w_{12}^{(2)}\gamma x_2,\\
  \frac{\partial f}{\partial b_{1}^{(1)}}(x_1,x_2;\theta)&=\alpha w_{11}^{(2)}\beta, \quad&
  \frac{\partial f}{\partial b_{2}^{(1)}}(x_1,x_2;\theta)&=\alpha w_{12}^{(2)}\gamma, \\
  \frac{\partial f}{\partial w_{11}^{(2)}}(x_1,x_2;\theta)&=\alpha g(w_{11}^{(1)}x_{1}+w_{12}^{(1)}x_2 + b_1^{(1)}),&&\\
  \frac{\partial f}{\partial w_{12}^{(2)}}(x_1,x_2;\theta)&=\alpha g(w_{21}^{(1)}x_{1}+w_{22}^{(1)}x_2 + b_2^{(1)}),&\quad
  \frac{\partial f}{\partial b_{1}^{(2)}}(x_1,x_2;\theta)&=\alpha.
\end{alignat*}

Como podemos observar, algunos de los factores de las parciales se repiten en varias de ellas, de forma que se ahorrarán muchos cálculos innecesarios si no se repiten. Este hecho se hace aún más evidente conforme se añaden capas a la red y unidades a cada capa. Por ello, el algoritmo de propagación hacia atrás permite optimizar el cálculo del gradiente mediante varios pasos intermedios para evitar cálculos repetidos.

\end{example}

En el algoritmo \ref{alg:backprop} se describe \emph{backprop} paso a
paso. Es fácil comprobar que aplicando esta técnica al ejemplo anterior
podemos evaluar las parciales que se han deducido sin repetir cálculos
costosos.

\begin{algorithm}
\caption{Propagación hacia atrás}
\label{alg:backprop}
\begin{algorithmic}
\STATE{Tras la propagación hacia adelante, calcular el gradiente de la capa de salida:}
\STATE{$d\gets \nabla_{y}J(y,y^{*};\theta)=\nabla_{y}L(y,y^{*})$}
  \FOR{$k=l,\dots,1$}
  \STATE{Aplicar la regla de la cadena a la función de activación ($\odot$ denota producto componente a componente):}
  \STATE{$d\gets \nabla_{z^{(k)}}J=d\odot g'(z^{i})$}
  \STATE{Calcular gradientes en los pesos y sesgos (incluyendo el término de regularización si es necesario):}
  \STATE{$\nabla_{b^{(k)}}J=d+\lambda\nabla_{b^{(k)}}\Omega(\theta)$}
  \STATE{$\nabla_{W^{(k)}}J=d\Tr{(h^{(k-1)})}+\lambda\nabla_{W^{(k)}}\Omega(\theta)$}
  \STATE{Propagar el gradiente hacia la capa oculta anterior:}
  \STATE{$d\gets \nabla_{h^{(k-1)}}J=\Tr{(W^{(k)})}d$}
  \ENDFOR
\end{algorithmic}
\end{algorithm}

\section{Optimización en Deep
Learning}\label{optimizaciuxf3n-en-deep-learning}

\label{sec:dl-opt}

\subsection{Gradiente descendiente estocástico
(SGD)}\label{gradiente-descendiente-estocuxe1stico-sgd}

En aprendizaje automático, y especialmente en Deep Learning, es común
utilizar conjuntos de datos con un gran número de instancias, para
favorecer la capacidad de generalización de los modelos producidos por
algoritmos. Esto provoca que el coste computacional de calcular cada
paso de un gradiente descendente haga inviable su uso. Sin embargo, se
puede utilizar una aproximación estocástica al algoritmo denominada
gradiente descendiente estocástico (\emph{Stochastic Gradient Descent},
SGD). En esta versión de gradiente descendiente se asienta la mayor
parte del desarrollo del Deep Learning en la actualidad.

Al ser una aproximación estocástica, SGD calcula un estimador del
gradiente de la función objetivo a partir de un número reducido de
muestras. Se describe en el algoritmo \ref{alg:sgd}.

\begin{algorithm}
\caption{Gradiente descendiente estocástico, iteración $k$-ésima}
\label{alg:sgd}
\begin{algorithmic}
  \REQUIRE{Tasa de aprendizaje $\varepsilon_k$}
  \REQUIRE{Parámetro inicial $\theta$}
  \WHILE{no se alcanza criterio de parada}
  \STATE{Escoger un minilote de $m$ instancias del conjunto de entrenamiento $x^{(1)},\dots,x^{(m)}$ con correspondientes objetivos $y^{(i)}$}
  \STATE{Calcular estimador del gradiente: $\hat g\gets \frac 1 m \nabla \sum_i L(f(x^{(i)}; \theta),y^{(i)})$}
  \STATE{Actualizar parámetro: $\theta\gets\theta - \varepsilon_k\hat g$}
  \ENDWHILE
\end{algorithmic}
\end{algorithm}

\subsection{Variantes de SGD}\label{variantes-de-sgd}

\subsubsection{SGD con momento}\label{sgd-con-momento}

El momento es un término adicional que fuerza a que SGD varíe menos la
dirección de una iteración a otra. De esta forma, el zigzagueo
característico de GD, también presente en SGD, se atenúa. Esta versión
se describe en el algoritmo \ref{alg:sgdm}.

\begin{algorithm}
\caption{Gradiente descendiente estocástico con momento}
\label{alg:sgdm}
\begin{algorithmic}
  \REQUIRE{Tasa de aprendizaje $\varepsilon$, momento $\alpha$}
  \REQUIRE{Parámetro inicial $\theta$, velocidad inicial $v$}
  \WHILE{no se alcanza criterio de parada}
  \STATE{Escoger un minilote de $m$ instancias del conjunto de entrenamiento $x^{(1)},\dots,x^{(m)}$ con correspondientes objetivos $y^{(i)}$}
  \STATE{Calcular estimador del gradiente: $\hat g\gets \frac 1 m \nabla \sum_i L(f(x^{(i)}; \theta),y^{(i)})$}
  \STATE{Actualizar la velocidad: $v\gets\alpha v - \varepsilon \hat g$}
  \STATE{Actualizar parámetro: $\theta\gets\theta + v$}
  \ENDWHILE
\end{algorithmic}
\end{algorithm}

\subsubsection{AdaGrad}\label{adagrad}

Adagrad \autocite{adagrad} es una versión adaptativa de SGD, en el
sentido de que varía los parámetros de forma inversamente proporcional a
la raíz cuadrada de la suma de los cuadrados de los valores anteriores.
Así, en lugar de decrementar la tasa de aprendizaje de igual forma para
todos los parámetros, la decrementa más rápido en los parámetros que
tienen mayores derivadas parciales. Como resultado adicional, en las
zonas de menor pendiente del espacio de parámetros el algoritmo progresa
más rápidamente que SGD. Se describe en el algoritmo \ref{alg:adagrad}.

\begin{algorithm}
\caption{Adagrad}
\label{alg:adagrad}
\textbf{Notación:} $\odot$ es el producto componente a componente, $\sqrt{.}$ es la raíz cuadrada componente a componente y la división por $\frac{1}{\delta + \sqrt r}$ se realiza componente a componente.
\begin{algorithmic}
  \REQUIRE{Tasa de aprendizaje $\varepsilon$, constante pequeña $\delta$}
  \REQUIRE{Parámetro inicial $\theta$}
  \STATE{Inicializar: $r\gets 0$}
  \WHILE{no se alcanza criterio de parada}
  \STATE{Escoger un minilote de $m$ instancias del conjunto de entrenamiento $x^{(1)},\dots,x^{(m)}$ con correspondientes objetivos $y^{(i)}$}
  \STATE{Calcular estimador del gradiente: $\hat g\gets \frac 1 m \nabla \sum_i L(f(x^{(i)}; \theta),y^{(i)})$}
  \STATE{Acumular cuadrado del gradiente: $r\gets r + \hat g\odot \hat g$}
  \STATE{Calcular actualización: $\Delta\theta\gets - \frac{\varepsilon}{\delta + \sqrt{r}}\odot \hat g$}
  \STATE{Actualizar parámetro: $\theta\gets\theta + \Delta\theta$}
  \ENDWHILE
\end{algorithmic}
\end{algorithm}

\subsubsection{RMSProp}\label{rmsprop}

El algoritmo RMSProp, detallado en el algoritmo \ref{alg:rmsprop},
sustituye la acumulación de gradientes de AdaGrad por una media
exponencial, de forma que tenga mejor comportamiento al optimizar
funciones no convexas.

\begin{algorithm}
\caption{RMSProp}
\label{alg:rmsprop}
\textbf{Notación:} De nuevo, las operaciones $\odot$, raíz cuadrada y división se realizan componente a componente.
\begin{algorithmic}
  \REQUIRE{Tasa de aprendizaje $\varepsilon$, constante pequeña $\delta$}
  \REQUIRE{Tasa de decaimiento $\rho$}
  \REQUIRE{Parámetro inicial $\theta$}
  \STATE{Inicializar: $r\gets 0$}
  \WHILE{no se alcanza criterio de parada}
  \STATE{Escoger un minilote de $m$ instancias del conjunto de entrenamiento $x^{(1)},\dots,x^{(m)}$ con correspondientes objetivos $y^{(i)}$}
  \STATE{Calcular estimador del gradiente: $\hat g\gets \frac 1 m \nabla \sum_i L(f(x^{(i)}; \theta),y^{(i)})$}
  \STATE{Acumular cuadrado del gradiente: $r\gets \rho r + (1 - \rho) \hat g\odot \hat g$}
  \STATE{Calcular actualización: $\Delta\theta\gets - \frac{\varepsilon}{\sqrt{\delta + r}}\odot \hat g$}
  \STATE{Actualizar parámetro: $\theta\gets\theta + \Delta\theta$}
  \ENDWHILE
\end{algorithmic}
\end{algorithm}

\subsubsection{Adam}\label{adam}

Adam es un algoritmo que también adapta la tasa de aprendizaje y además
introduce un momento adaptativo, se puede considerar una combinación de
RMSProp con momento. Se describe en el algoritmo \ref{alg:adam}.

\begin{algorithm}
\caption{Adam}
\label{alg:adam}
\begin{algorithmic}
  \REQUIRE{Tasa de aprendizaje $\varepsilon$, constante pequeña $\delta$}
  \REQUIRE{Tasas de decaimiento exponencial $\rho_{1}, \rho_2\in[0,1[$}
  \REQUIRE{Parámetro inicial $\theta$}
  \STATE{Inicializar: $s\gets 0, r\gets 0, t\gets 0$}
  \WHILE{no se alcanza criterio de parada}
  \STATE{Escoger un minilote de $m$ instancias del conjunto de entrenamiento $x^{(1)},\dots,x^{(m)}$ con correspondientes objetivos $y^{(i)}$}
  \STATE{Calcular estimador del gradiente: $\hat g\gets \frac 1 m \nabla \sum_i L(f(x^{(i)}; \theta),y^{(i)})$}
  \STATE{Incrementar tiempo: $t\gets t + 1$}
  \STATE{Actualizar estimador sesgado del 1er momento: $s\gets \rho_1 s + (1 - \rho_1)\hat g$}
  \STATE{Actualizar estimador sesgado del 2º momento: $r\gets \rho_2 s + (1 - \rho_2)\hat g\odot \hat g$}
  \STATE{Corregir sesgos: $\hat s\gets\frac{s}{1 - \rho_1^t},\ \hat r\gets\frac{r}{1 - \rho_2^t}$}
  \STATE{Calcular actualización: $\Delta\theta\gets - \frac{\varepsilon}{\delta + \sqrt{\hat r}}\hat s$ (operaciones componente a componente)}
  \STATE{Actualizar parámetro: $\theta\gets\theta + \Delta\theta$}
  \ENDWHILE
\end{algorithmic}
\end{algorithm}

\section{Estructuras profundas no
supervisadas}\label{estructuras-profundas-no-supervisadas}

\subsection{Máquina de Boltzmann restringidas
(RBM)}\label{muxe1quina-de-boltzmann-restringidas-rbm}

\subsection{Autoencoder}\label{autoencoder}

\begin{figure}[hbtp]
  \centering
\begin{tikzpicture}[scale=0.2]
\tikzstyle{every node}+=[inner sep=0pt]
\draw [black] (21.5,-13.4) circle (3);
\draw [black] (21.5,-23.4) circle (3);
\draw [black] (21.5,-33.1) circle (3);
\draw [black] (36.4,-23.4) circle (3);
\draw [black] (36.4,-33.1) circle (3);
\draw [black] (50.6,-23.4) circle (3);
\draw [black] (21.5,-43) circle (3);
\draw [black] (50.6,-33.1) circle (3);
\draw [black] (50.6,-43) circle (3);
\draw [black] (50.6,-13.4) circle (3);
\draw [black] (24.5,-33.1) -- (33.4,-33.1);
\fill [black] (33.4,-33.1) -- (32.6,-32.6) -- (32.6,-33.6);
\draw [black] (24.01,-31.46) -- (33.89,-25.04);
\fill [black] (33.89,-25.04) -- (32.94,-25.05) -- (33.49,-25.89);
\draw [black] (24.5,-23.4) -- (33.4,-23.4);
\fill [black] (33.4,-23.4) -- (32.6,-22.9) -- (32.6,-23.9);
\draw [black] (24.01,-25.04) -- (33.89,-31.46);
\fill [black] (33.89,-31.46) -- (33.49,-30.61) -- (32.94,-31.45);
\draw [black] (23.99,-15.07) -- (33.91,-21.73);
\fill [black] (33.91,-21.73) -- (33.52,-20.87) -- (32.97,-21.7);
\draw [black] (23.31,-15.79) -- (34.59,-30.71);
\fill [black] (34.59,-30.71) -- (34.51,-29.77) -- (33.71,-30.37);
\draw [black] (39.4,-23.4) -- (47.6,-23.4);
\fill [black] (47.6,-23.4) -- (46.8,-22.9) -- (46.8,-23.9);
\draw [black] (38.88,-31.41) -- (48.12,-25.09);
\fill [black] (48.12,-25.09) -- (47.18,-25.13) -- (47.74,-25.96);
\draw [black] (24,-41.34) -- (33.9,-34.76);
\fill [black] (33.9,-34.76) -- (32.96,-34.79) -- (33.51,-35.62);
\draw [black] (23.32,-40.61) -- (34.58,-25.79);
\fill [black] (34.58,-25.79) -- (33.7,-26.12) -- (34.5,-26.73);
\draw [black] (38.85,-21.67) -- (48.15,-15.13);
\fill [black] (48.15,-15.13) -- (47.21,-15.18) -- (47.78,-16);
\draw [black] (38.88,-25.09) -- (48.12,-31.41);
\fill [black] (48.12,-31.41) -- (47.74,-30.54) -- (47.18,-31.37);
\draw [black] (38.16,-25.83) -- (48.84,-40.57);
\fill [black] (48.84,-40.57) -- (48.78,-39.63) -- (47.97,-40.22);
\draw [black] (38.15,-30.67) -- (48.85,-15.83);
\fill [black] (48.85,-15.83) -- (47.97,-16.19) -- (48.78,-16.78);
\draw [black] (39.4,-33.1) -- (47.6,-33.1);
\fill [black] (47.6,-33.1) -- (46.8,-32.6) -- (46.8,-33.6);
\draw [black] (38.86,-34.82) -- (48.14,-41.28);
\fill [black] (48.14,-41.28) -- (47.77,-40.42) -- (47.2,-41.24);
\end{tikzpicture}
\caption{\label{fig:autoencoder}Autoencoder de tres capas} 
\end{figure}

\textcite{hinton2006autoencoder}

\subsection{Entrenamiento de
autoencoders}\label{entrenamiento-de-autoencoders}

\subsubsection{Pre-entrenamiento}\label{pre-entrenamiento}

\subsubsection{Ajuste fino}\label{ajuste-fino}


\chapter{La herramienta Ruta}
\section{El lenguaje R}\label{introducciuxf3n-a-r}

\subsection{Introducción}

R \autocite{rlang} es un lenguaje de programación dirigido al tratamiento de datos, y
como tal proporciona estructuras de datos y funcionalidades básicas para
representar y tratar problemas de minería de datos. Es un proyecto GNU\footnote{GNU es un proyecto extenso que abarca distintos paquetes de software libre, R es uno de ellos: \url{https://www.gnu.org/}.}, y se considera una implementación alternativa del lenguaje S para estadística, desarrollado originalmente en Bell Labs.

Además, existe toda
una plataforma de paquetes para R denominada CRAN\footnote{\url{https://cran.r-project.org}}, que cuenta con
multitud de bibliotecas que facilitan tareas muy diversas, desde lectura y
visualización de datos hasta el propio procesamiento mediante distintos
algoritmos.

\subsection{Instalación}

Para utilizar el software desarrollado, será necesario instalar R junto con las utilidades de desarrollador del lenguaje. En general, hay disponibles paquetes binarios en los repositorios de las distribuciones más comunes de Linux, así como para Windows y macOS. Por ejemplo, para instalar R en distribuciones basadas en Debian como Ubuntu, ejecutaremos el siguiente comando:

\begin{verbatim}
sudo apt install r-base r-base-dev
\end{verbatim}

Los instaladores para Windows y macOS se pueden descargar desde el sitio web del proyecto R\footnote{\url{https://www.r-project.org/}}.

\subsection{Uso del lenguaje}

El lenguaje R se puede utilizar de forma interactiva mediante el REPL\footnote{Un REPL (\emph{read-eval-print-loop}) es un entorno de programación interactivo donde se evalúa el código línea a línea.} que se invoca con el comando \texttt{R}, mediante scripts que se pueden ejecutar con el programa \texttt{Rscript} o desde un IDE como RStudio \autocite{rstudio}.

R soporta la mayoría de tipos de dato básicos de cualquier otro lenguaje:
\begin{itemize}
\item \textit{logical}: valores lógicos entre \texttt{TRUE}, \texttt{FALSE} o \texttt{NA}
\item \textit{integer}: valores enteros
\item \textit{double}: valores reales de doble precisión (junto con \texttt{integer} forman el tipo \texttt{numeric})
\item \textit{complex}: números complejos
\item \textit{character}: cadenas de caracteres
\item \textit{raw}: datos binarios en bruto
\end{itemize}

Se caracteriza por el uso de algunas estructuras de datos: los vectores atómicos, las matrices, las listas y los \textit{data.frames}. Estos últimos son muy útiles para representar conjuntos de datos.

Las funciones son objetos de primera clase en R. Esto quiere decir que se pueden y suelen pasar como argumento a otras funciones, y devolver como valor de retorno. Esto permite el uso de funciones de orden superior clásicas del tipo \textit{map} o \textit{reduce}, que realizan operaciones simultáneamente sobre varios elementos de una estructura de datos.

R incorpora tres sistemas de orientación a objetos distintos: S3, S4 y \textit{Reference Classes} (RC). El primero de ellos es el más sencillo y el más utilizado, y es también el que se usa en el software desarrollado. Consiste simplemente en objetos tipo lista a los que se le ha aplicado un atributo de clase mediante la función \texttt{class}. Este atributo permite escoger un método cuando se realiza una llamada a una función genérica.

\begin{example}
  Para realizar una demostración sobre la orientación a objetos S3 de R, vamos a definir objetos de clase \texttt{animal} que implementen el método \texttt{print}. Este es ya una función genérica de R, luego la asociación del método que definamos con nuestros objetos será automática. Para crearlos, podemos usar una función que hará las veces de constructor:
  \begin{lstlisting}
animal <- function(nombre, sonido) {
  objeto <- list(
    nombre = nombre,
    sonido = sonido
  )
  class(objeto) <- "animal"
  return(objeto)
}
print.animal <- function(animal) {
  print(paste("Soy un", animal$nombre, "y hago", animal$sonido))
}
print(animal("gato", "miau"))
# => [1] "Soy un gato y hago miau"
print(animal("perro", "guau"))
# => [1] "Soy un perro y hago guau"
\end{lstlisting}

Nótese que en las líneas 12 y 14 del código anterior se llama a la función \texttt{print}, que al ser genérica busca el método correspondiente para los objetos de clase \texttt{animal}.
\end{example}

\section{La biblioteca MXNet}\label{sec:mxnet}
\subsection{Descripción}
MXNet \autocite{mxnet} es una biblioteca de algoritmos de Deep Learning, es decir, incluye la funcionalidad necesaria para construir estructuras de aprendizaje profundas, calcular los gradientes con propagación hacia atrás (sección \ref{sec:backprop}), entrenarlas con datos de entrada mediante los algoritmos de optimización estudiados en la sección \ref{sec:dl-opt} y realizar predicciones sobre nuevos datos.

Frente a otras bibliotecas similares como Tensorflow \autocite{tensorflow} o Theano \autocite{theano}, el motor de MXNet está escrito en el lenguaje C++, lo que reduce los tiempos de ejecución. Sin embargo, esto no limita su uso puesto que proporciona acceso a las funcionalidades mediante APIs para otros lenguajes como Python, Scala o R.

Esta biblioteca permite ejecutar los algoritmos de forma secuencial, distribuida o en dispositivos GPU. Además, proporciona dos mecánicas de programación diferenciadas: simbólica e imperativa. Por un lado, la programación simbólica permite diseñar un modelo de forma rápida sin necesidad de aportar los datos de entrada \textit{a priori} ni ejecutar los algoritmos de forma inmediata. Esto permite una programación más flexible, pudiendo acceder a los parámetros y modificarlos de forma desconectada de los cálculos costosos. Por otro lado, la programación imperativa facilita el control sobre los procesos de aprendizaje y la forma en que los datos se propagan por una red neuronal.

\subsection{Instalación}

Para disponer de MXNet en un ordenador y poder usarla desde R, es necesario compilar e instalar tanto la biblioteca como el paquete compañero para R. Las dependencias son las herramientas de compilación básicas (G++, GNU Make) y una implementación de la biblioteca de álgebra lineal BLAS. como OpenBLAS o ATLAS. Opcionalmente se puede instalar OpenCV para facilitar el tratamiento de imágenes. Tras esto, se ejecutan los siguientes comandos para descargar y compilar el software:

\begin{lstlisting}[language=sh,frame=none]
git clone --recursive https://github.com/dmlc/mxnet
cd mxnet
cp make/config.mk . # editar las entradas necesarias
make -j $(nproc)
sudo install -D lib/libmxnet.so /usr/lib/libmxnet.so
\end{lstlisting}
%$ <-- hack for emacs syntax highlighting

Si se desea soporte para cómputo sobre GPU con CUDA, se añadirán las opciones \texttt{USE\_CUDA=1} \texttt{USE\_CUDA\_PATH=/opt/cuda} \texttt{USE\_CUDNN=1} al archivo \texttt{config.mk}, utilizando el camino conveniente para la biblioteca CUDA.

Alternativamente, en sistemas basados en Arch Linux basta con instalar el paquete \texttt{mxnet}\footnote{\url{https://aur.archlinux.org/packages/mxnet/}} del repositorio de usuarios AUR.

Por último, para instalar el paquete compañero para R, se utilizan las siguientes órdenes de línea de comandos desde el directorio donde se ha clonado el repositorio:

\begin{lstlisting}[language=sh,frame=none]
make rpkg
R CMD INSTALL mxnet_current_r.tar.gz
\end{lstlisting}

\subsection{Uso de la biblioteca}
Para construir una estructura de aprendizaje profunda con MXNet, basta con comenzar con un símbolo que corresponderá a los datos de entrada, después especificar las capas que formarán la red y, por último, el tipo de salida deseada.

Después, para ajustar el modelo creado con un conjunto de entrenamiento, MXNet proporciona algunas utilidades de alto nivel y otras más cercanas al cómputo paso a paso de las propagaciones hacia adelante y hacia atrás. De las primeras podemos destacar \texttt{mx.model.FeedForward.create} y de las últimas \texttt{mx.exec.backward} y \texttt{mx.exec.forward}.

\begin{example}

  En este ejemplo vamos a construir una red prealimentada sencilla que permitirá aproximar el cálculo de la hipotenusa de un triángulo rectángulo a partir de las longitudes de los catetos.

  Primero, comenzamos construyendo la red. Creamos la variable simbólica que contendrá los datos y la enlazamos con dos capas ocultas de 2 y 10 unidades respectivamente, y con la capa de salida que aproximará la hipotenusa, evaluará la pérdida y permitirá realizar el aprendizaje.
  
\begin{lstlisting}[language=R,frame=none]
library(mxnet)
red <- mx.symbol.Variable("data")
red <- mx.symbol.FullyConnected(red, num_hidden = 2)
red <- mx.symbol.Activation(red, act_type = "relu")
red <- mx.symbol.FullyConnected(red, num_hidden = 10)
red <- mx.symbol.Activation(red, act_type = "relu")
red <- mx.symbol.FullyConnected(red, num_hidden = 1)
red <- mx.symbol.LinearRegressionOutput(red)
\end{lstlisting}
  
  Ahora, creamos los datos de entrada: aleatoriamente escogemos catetos y después calculamos la hipotenusa en la variable \texttt{label}. Separamos en un conjunto para entrenamiento y otro para test. De forma similar, podríamos también proceder con una validación cruzada.
  
\begin{lstlisting}[language=R,frame=none]
set.seed(42)
mx.set.seed(42)

input <- data.frame(
  cat1 = round(runif(100, min = 1, max = 10)),
  cat2 = round(runif(100, min = 1, max = 10)))
x = t(data.matrix(input))
label <- sqrt(input$cat1 ** 2 + input$cat2 ** 2)
train_x <- x[,1:74]
train_y <- label[1:74]
test_x <- x[,75:100]
test_y <- label[75:100]
\end{lstlisting}

Por último, entrenamos el modelo que creamos anteriormente, escogiendo los parámetros del proceso de aprendizaje, en particular el optimizador SGD explicado en la sección \ref{sec:sgd}, y obtenemos las predicciones sobre el conjunto de test.
  
\begin{lstlisting}[language=R,frame=none]
model <- mx.model.FeedForward.create(
  symbol = red,
  X = train_x,
  y = train_y,
  num.round = 240,
  array.layout = "colmajor",
  optimizer = "sgd",
  learning.rate = 0.015,
  momentum = 0.2,
  eval.metric = mx.metric.rmse
)
# => Train-rmse=0.693727073714297
predict(model, test_x)
\end{lstlisting}
  
\end{example}


\section{Introducción a Ruta}\label{el-paquete-ruta}

Se ha desarrollado un paquete software, denominado Ruta, con el
objetivo de proporcionar acceso a diversas estructuras de aprendizaje profundo no supervisado de forma muy sencilla. El paquete se ha escrito en el lenguaje R utilizando los recursos de la biblioteca MXNet.

\subsection{Desarrollo y metodologías}\label{metodologuxeda-de-desarrollo}

Al comienzo del desarrollo del software, se han analizado los objetivos perseguidos y se han escogido las tecnologías que nos facilitarían su consecución. En concreto, se ha elegido el lenguaje R por varias razones. Por un lado apenas dispone de herramientas de Deep Learning cómodas y consolidadas, luego este nuevo software no es redundante con nada existente. Además, R es un lenguaje que aporta muchas facilidades para la visualización de datos, lo cual es beneficioso ya que parte del software desarrollado se dedica a eso. También se eligió la biblioteca MXNet tras comparar con varias de las competidoras en el mercado en ese momento: Tensorflow se descartó por ser relativamente más lenta que el resto, y Theano y Caffe no disponían de API para R, mientras que otras librerías no estaban tan desarrolladas y no aportaban la funcionalidad que necesitábamos.

La metodología de desarrollo ha sido de tipo ágil, es decir, se ha desarrollado partiendo de un prototipo funcional, alterando y aumentando sus funcionalidades según se ha ido requiriendo. La documentación\footnote{Se puede consultar desde R mediante \texttt{?} delante del nombre de la función. Por ejemplo, \texttt{?ruta.makeTask}.} es exhaustiva en el sentido de que cubre toda la funcionalidad disponible para el usuario, pero no es excesiva. Además, progresivamente se ha ido adaptando el desarrollo según las nuevas necesidades o los obstáculos encontrados.

Un prototipo preliminar de esta herramienta, con funcionalidades básicas de entrenamiento de autoencoders y visualizaciones sencillas, se implementó bajo el nombre \emph{dlvisR} y se presentó en el congreso nacional CAEPIA 2016 \autocite{charte2016dlvisr}.

Puesto que se trata de un software científico que se sostiene sobre varias piezas de software libre, se ha decidido que también esta herramienta será de código abierto y libre. Para asegurar que se mantiene libre, además, se ha utilizado la licencia \textit{GNU General Public License} (GPL) 3.0, que requiere que las modificaciones que se realicen sobre el código del software se liberen bajo la misma licencia\footnote{Texto de la licencia disponible en \url{https://www.gnu.org/licenses/gpl.html}.}. Así, el código del software está publicado en el repositorio fdavidcl/ruta de GitHub\footnote{\url{https://github.com/fdavidcl/ruta/}}.

Para asistir a la comprobación del software completo durante su desarrollo, se ha utilizado un sistema de integración continua, es decir, automatización de \textit{builds} y de tests. Este sistema realiza, a cada avance publicado en GitHub, una comprobación mediante el programa \texttt{R CMD check}.

\subsection{Instalación}

Para instalar la última versión estable de Ruta y el paquete compañero Rutavis, basta con utilizar el conocido paquete \texttt{devtools} \autocite{devtools} para realizar automáticamente la descarga y la instalación. Desde la consola de R ejecutamos:
\begin{lstlisting}[numbers=none]
devtools::install_github("fdavidcl/ruta")
devtools::install_github("fdavidcl/rutavis")
\end{lstlisting}

Una vez instalado, para cargarlos utilizamos las órdenes siguientes:
\begin{lstlisting}[numbers=none]
library(ruta)
library(rutavis)
\end{lstlisting}


\section{Componentes del paquete y uso}\label{componentes-del-paquete}

\subsection{Estructura}

El software se estructura como un paquete convencional de R, siguiendo la jerarquía de directorios siguiente:
\begin{itemize}
\item \texttt{man/}
  \begin{itemize}
  \item archivos de documentación en formato \texttt{.Rd}.
  \end{itemize}
\item \texttt{R/}
  \begin{itemize}
  \item \texttt{autoencoder\_learner.R}: implementa la generación de redes profundas tipo autoencoder.
  \item \texttt{autoencoder\_model.R}: incluye el entrenamiento de autoencoders y las herramientas de análisis de modelos.
  \item \texttt{classes.R}: incluye las clases utilizadas a lo largo del paquete.
  \item \texttt{learner.R}: incluye las funcionalidades comunes a los algoritmos de aprendizaje.
  \item \texttt{model.R}: implementa funcionalidades comunes a los modelos.
  \item \texttt{rbm\_learner.R}: implementa la generación de máquinas de Boltzmann restringidas.
  \item \texttt{rbm\_model.R}: incluye el entrenamiento de RBMs.
  \item \texttt{task.R}:  contiene las funcionalidades comunes a las tareas.
  \item \texttt{unsupervised\_task.R}: implementa la gestión de tareas no supervisadas. 
  \item \texttt{util.R}: incluye utilidades adicionales.
  \end{itemize}
\item \texttt{DESCRIPTION}: indica metadatos del paquete, como nombre, versión y dependencias.
\item \texttt{LICENSE}: contiene el texto de la licencia libre, en este caso GPL 3.0.
\item \texttt{NAMESPACE}: da información sobre los nombres de funciones exportadas e importadas.
\end{itemize}

\subsection{Funcionalidad}

Las funcionalidades del paquete se han dividido en tres categorías:
\begin{itemize}
\item Tareas
\item Algoritmos de aprendizaje
\item Modelos entrenados
\end{itemize}
Aquí se ha utilizado como guía la arquitectura del paquete \texttt{mlr} \autocite{mlr} dirigido a aprendizaje automático en general.

Por un lado, las tareas representan conjuntos de datos de los que se desea aprender un modelo. Para ello, se construirán representaciones de los algoritmos de aprendizaje, ajustadas con diversos parámetros, que después se podrán entrenar y generar un modelo. Dicho modelo se podrá estudiar para obtener información sobre nuevos datos o sobre los aprendidos.

\subsection{Tareas}

Una tarea de aprendizaje se compone de un conjunto de datos y varios metadatos que aportan información acerca del mismo. En Ruta se pueden crear tareas genéricas con la función \texttt{ruta.makeTask} y tareas de aprendizaje no supervisado con \texttt{ruta.makeUnsupervisedTask}.

\begin{example}
  Construyamos una tarea de aprendizaje no supervisado a partir del conocido conjunto de datos Iris \autocite{fisher1936iris}. Para ello, será necesario cargar el conjunto y posteriormente generaremos una tarea de Ruta, indicando la columna en la que se encuentra la clase, y podremos obtener información sobre ella:
  \begin{lstlisting}
data(iris)
task <- ruta.makeUnsupervisedTask("iris", data = iris, cl = 5)
print(task)
# ruta Task: iris
# Type: unsupervised
# Instances: 150
# Features: 5
# Has class: Yes (5)
  \end{lstlisting}
\end{example}

\subsection{Algoritmos de aprendizaje}

En el momento actual del desarrollo, Ruta cuenta con una implementación de autoencoders basada en la biblioteca MXNet y una implementación básica de RBMs. Para utilizarlas, se dispone de la función \texttt{ruta.makeLearner} que da acceso a las técnicas de aprendizaje implementadas. Además, para generar modelos a partir de algoritmos y tareas, basta con utilizar el método \texttt{train} que se ha especializado para los objetos de clase \texttt{``rutaAutoencoder''}.

\begin{example}\label{ex:ruta2}
  Utilizamos la función mencionada para crear un objeto que represente un autoencoder de 5 capas de 4, 10, 2, 10 y 4 unidades respectivamente, con activaciones ReLU.

  Lo entrenamos con la tarea que creamos anteriormente, seleccionando el optimizador Adam (\autoref{alg:adam}) con una tasa de aprendizaje de $0.005$:
  \begin{lstlisting}
ae <- ruta.makeLearner("autoencoder",
                       hidden = c(4, 10, 2, 10, 4),
                       activation = "relu")
print(ae)
# ruta Learner
# Type: Autoencoder
# Backend: mxnet
# Sparse: No

model <- train(ae, task,
               epochs = 500,
               learning.rate = 0.005,
               optimizer = "adam")
  \end{lstlisting}
\end{example}

Destacamos algunos parámetros interesantes en el ejemplo anterior. Primero, \texttt{activation}, que indica la función de activación que llevarán las unidades de la red. Las funciones de activación disponibles son:
\begin{itemize}
\item Ninguna (unidad lineal): \texttt{NULL}
\item ReLU: \texttt{``relu''}
\item Función logística: \texttt{``sigmoid''}
\item Función \emph{softplus}: \texttt{``softrelu''}
\item Función tangente hiperbólica: \texttt{``tanh''}
\item Tipo \emph{leaky} ReLU: \texttt{``leaky''}, \texttt{``elu''}, \texttt{``prelu''}, \texttt{``rrelu''}
\end{itemize}

Por otra parte, el parámetro \texttt{optimizer} permite escoger el algoritmo de optimización con el que se entrenará. Los optimizadores disponibles son:
\begin{itemize}
\item Gradiente descendiente estocástico (\autoref{alg:sgd}): \texttt{``sgd''}
\item Adagrad (\autoref{alg:adagrad}): \texttt{``adagrad''}
\item RMSProp (\autoref{alg:rmsprop}): \texttt{``rmsprop''}
\item Adam (\autoref{alg:adam}): \texttt{``adam''}
\end{itemize}
Cada uno de ellos acepta además los parámetros adicionales propios del algoritmo correspondiente.

\subsection{Modelos entrenados}

Una vez se ha obtenido  un modelo entrenado, se pueden realizar distintas operaciones sobre él. Una de las más interesantes es obtener las codificaciones para unos datos dados, es decir, las salidas de la capa interna del autoencoder. Para ello contamos con la función \texttt{ruta.deepFeatures}. Adicionalmente, la función \texttt{ruta.layerOutputs} permite extraer las salidas de cualquier capa de la red. Por último, la implementación del método \texttt{predict} para objetos de clase \texttt{``rutaModel''} facilita las salidas de la última capa del autoencoder.

\begin{example}
Ahora tomamos el modelo aprendido en el \autoref{ex:ruta2} y obtenemos, para los mismos datos con los que se entrenó, las codificaciones en la capa interna:
  \begin{lstlisting}
deepf <- ruta.deepFeatures(model, task)
# Extracting layer aelayer3act_output (output #13)
# Setting arguments up to aelayer3_bias (arg #6)
  \end{lstlisting}
\end{example}

Para conseguir extraer estas características internas, ha sido necesaria la implementación de una funcionalidad ausente en la API de MXNet para R: la predicción de capas internas de una red. Dicha implementación se encuentra también incluida en el paquete como la función \texttt{predictPartial}. Esta función trabaja a bajo nivel, controlando la propagación de datos a través de cada capa.

\section{Visualización con Rutavis}\label{sec:rutavis}

Una herramienta complementaria a \texttt{ruta} es \texttt{rutavis}, una
aplicación con interfaz de usuario web que permite componer
visualizaciones de forma interactiva y compararlas.

\subsection{Estructura}

De nuevo, este software se estructura como un paquete convencional de R, siguiendo una jerarquía similar a Ruta pero con algunas diferencias:
\begin{itemize}
\item \texttt{inst/}
  \begin{itemize}
  \item \texttt{shiny/}
  \begin{itemize}
  \item \texttt{www/}
  \begin{itemize}
  \item archivos adicionales para la interfaz web (CSS y JavaScript).
  \end{itemize}
  \item \texttt{template.html}: plantilla HTML que compone la web.
  \item \texttt{server.R}: script que ejecuta el servidor web.
  \item \texttt{ui.R}: script que compone la interfaz del cliente mediante la plantilla.
  \end{itemize}
  \end{itemize}
\item \texttt{man/}
  \begin{itemize}
  \item archivos de documentación en formato \texttt{.Rd}.
  \end{itemize}
\item \texttt{R/}
  \begin{itemize}
  \item \texttt{gui.R}: implementa el lanzamiento del servidor web.
  \item \texttt{plots.R}: implementa distintos tipos de gráficos para representar modelos.
  \end{itemize}
\item \texttt{DESCRIPTION}: indica metadatos del paquete, como nombre, versión y dependencias.
\item \texttt{LICENSE}: contiene el texto de la licencia libre, en este caso GPL 3.0.
\item \texttt{NAMESPACE}: da información sobre los nombres de funciones exportadas e importadas.
\end{itemize}

\subsection{Funcionalidad}

Rutavis es un paquete totalmente dependiente de Ruta, centrado exclusivamente en la generación de gráficos a partir de modelos entrenados. Cumple dos responsabilidades principales: por una parte, la gestión de visualizaciones que ayuden a entender el comportamiento de las técnicas no supervisadas y de los modelos que entrenan; por otra, una interfaz web que facilita al usuario la creación de dichas visualizaciones.

Los gráficos generados se muestran mediante la herramienta de visualización Plotly \autocite{plotly}, que tiene una amplia gama de gráficas y una extensa documentación. Cuenta también con un paquete homónimo para R.

Además, para la creación del servidor web y la aplicación web desde R, se ha aprovechado el paquete Shiny \autocite{shiny}, que proporciona diversas utilidades para el intercambio de datos entre el cliente y el servidor. Se ha utilizado el \emph{framework} CSS Bulma\footnote{\url{http://bulma.io}} para dar estructura y estilo a los elementos de la página. La interactividad con el usuario se ha programado en JavaScript en el cliente, que envía los datos al servidor y gestiona las respuestas.

\subsection{Uso de la aplicación}\label{uso-de-la-aplicaciuxf3n}

Con Rutavis se pueden crear gráficos sin necesidad de hacer uso de la interfaz web, mediante la implementación de la función genérica \texttt{plot} para objetos de clase \texttt{``rutaModel''}.

\begin{example}
  Con una tarea y un modelo como los utilizados anteriormente para el conjunto de datos iris, vamos a generar un gráfico que muestre las salidas codificadas mediante la capa intermedia del autoencoder. Utilizamos el método \texttt{plot.rutaModel} y le pasamos las mismas instancias con las que se entrenó:
\begin{lstlisting}[numbers=none]
plot(model, task)
\end{lstlisting}
\begin{figure}[hbtp]
  \centering
  \includegraphics[width=0.8\textwidth]{images/rutavis_iris.png}
  \caption{\label{fig:irisplot}Gráfico de la codificación de los datos de entrada mediante la capa interna del autoencoder}
\end{figure}

La salida será un gráfico del estilo del de la \autoref{fig:irisplot}. 
Como se puede observar, el autoencoder consigue comprimir gran parte de la información de las 4 variables de Iris en solamente 2. Pese a no haber utilizado información sobre la clase, la representación bidimensional del conjunto de datos mantiene el bajo solapamiento entre clases.
\end{example}

Para lanzar la interfaz de usuario web, simplemente hay que ejecutar la siguiente llamada:
\begin{lstlisting}[numbers=none]
rutavis::ruta.gui()
\end{lstlisting}

Se abrirá una pestaña de navegador con la interfaz de usuario, inicialmente en la pantalla mostrada en la \autoref{fig:rutavis-welcome}, que nos permitirá crear una nueva visualización. Se pueden gestionar simultáneamente varias visualizaciones, de forma que se puedan estudiarse una a una y modificar sus parámetros, ver comparativamente en formato mosaico, o listar en estilo \emph{notebook} junto a los parámetros establecidos.

\begin{figure}[hbtp]
  \centering
  \fbox{\includegraphics[width=0.8\textwidth]{images/rutavis_welcome}}
  \caption[Interfaz gráfica de Rutavis]{Pantalla de bienvenida de Rutavis}
  \label{fig:rutavis-welcome}
\end{figure}

Vamos a realizar paso a paso un entrenamiento de dos modelos basados en autoencoders para el conjunto de datos WDBC (\emph{Wisconsin Diagnostic Breast Cancer}) obtenido de \autocite{Lichman:2013}. El conjunto representa un problema de clasificación binaria, donde a partir de características médicas se pretende predecir si un tumor es benigno o maligno.

Comenzamos lanzando la interfaz web y seleccionando la opción \emph{New visualization}. Se abrirá una vista compuesta por un panel de ajustes y un espacio para los gráficos. Cargamos el conjunto de datos mediante la opción \emph{Upload new dataset} del panel. Seleccionamos la característica que corresponde con la clase de la lista \emph{Class attribute}.

Para configurar un modelo de aprendizaje, escogemos los parámetros deseados. Como se puede observar en el ejemplo de la \autoref{fig:rutavis2}, utilizamos capas con unidades de tipo \emph{leaky} ReLU, una codificación en 3 variables, un entrenamiento con AdaGrad de 200 épocas ajustando la tasa de aprendizaje a $0.02$. En la \autoref{fig:rutavis1} se muestra la vista de visualización única con el gráfico resultante, que corresponde a la salida de la capa interna del autoencoder respecto a los datos de entrada (los puntos verdes corresponden a tumores benignos y los azules a malignos). Se trata de un gráfico interactivo, sobre el que podemos arrastrar el ratón para mover los ejes, acercar o alejar la vista, y guardar en formato PNG.

\begin{figure}[hbtp]
  \centering
  \fbox{\includegraphics[width=0.4\textwidth]{images/rutavis_params}}
  \caption[Interfaz gráfica de Rutavis]{El panel de ajuste de parámetros en detalle}
  \label{fig:rutavis2}
\end{figure}

\begin{figure}[hbtp]
  \centering
  \fbox{\includegraphics[width=\textwidth]{images/rutavis_plot2}}
  \caption[Interfaz gráfica de Rutavis]{Interfaz gráfica de usuario de Rutavis. Arriba, la barra de pestañas y modos de visualización. A la izquierda, ajuste de parámetros. En el centro, visualización de las gráficas generadas}
  \label{fig:rutavis1}
\end{figure}

Añadimos un segundo entorno de visualización mediante el botón \emph{+} de la barra superior de pestañas. Entrenamos un modelo similar, en este caso con unidades con función de activación tangente hiperbólica y distintos parámetros de épocas y tasa de aprendizaje. Seleccionamos la vista de mosaico para comparar ambos resultados, como se muestra en la \autoref{fig:rutavis3}. La vista estilo \emph{notebook} es similar a esta, pero apilando verticalmente todas las visualizaciones añadidas.

\begin{figure}[hbtp]
  \centering
  \fbox{\includegraphics[width=\textwidth]{images/rutavis_compare}}
  \caption[Modo mosaico de Rutavis]{Comparación de dos visualizaciones en el modo mosaico de Rutavis}
  \label{fig:rutavis3}
\end{figure}

\part{Conclusiones}

\chapter{Conclusiones}

% \part{Some Kind of Manual}
% \include{Chapters/Chapter01}
% \cleardoublepage
% \ctparttext{You can put some informational part preamble text here. 
% Illo principalmente su nos. Non message \emph{occidental} angloromanic
% da. Debitas effortio simplificate sia se, auxiliar summarios da que,
% se avantiate publicationes via. Pan in terra summarios, capital
% interlingua se que. Al via multo esser specimen, campo responder que
% da. Le usate medical addresses pro, europa origine sanctificate nos se.}
% \part{The Showcase}
% \include{Chapters/Chapter02}
% %\addtocontents{toc}{\protect\clearpage} % <--- just debug stuff, ignore
% \include{Chapters/Chapter03}
% \include{multiToC} % <--- just debug stuff, ignore for your documents

% ********************************************************************
% Backmatter
%*******************************************************
\appendix
%\renewcommand{\thechapter}{\alph{chapter}}
\cleardoublepage
\part{Apéndice}
\chapter{Recordatorio de Teoría de la Probabilidad}
A lo largo de este capítulo se definen conceptos básicos de la teoría de la probabilidad, que serán utilizados en el resto del trabajo. También se hace mención a unas propiedades de la convergencia que nos permiten después demostrar un resultado técnico que motiva el problema de reducción de la dimensionalidad, estudiado con más detalle en la \autoref{sec:red-dim}.  La fuente principal del capítulo es \textcite[capítulo 3]{goodfellow2016}.

\section{Recordatorio de conceptos}\label{conceptos}

El objetivo de la probabilidad es modelar y trabajar con incertidumbre.
Dicha incertidumbre puede provenir de diversas fuentes, entre ellas:

\begin{itemize}
\tightlist
\item
  Estocasticidad del sistema modelado (e.g.~mecánica cuántica,
  escenarios hipotéticos con aleatoriedad, etc.).
\item
  Falta de observabilidad: los sistemas deterministas se muestran
  aparentemente estocásticos cuando no se pueden observar todas las
  variables que los afectan.
\item
  Modelización incompleta: el uso de un modelo que descarta parte de la
  información observada (un modelo simple pero incompleto puede ser más
  útil que uno absolutamente preciso).
\end{itemize}

En el ámbito de estudio de este trabajo, el del aprendizaje automático, la
teoría de la probabilidad nos sirve para estudiar los algoritmos de
aprendizaje desde un punto de vista teórico y construir representaciones de los
modelos que aprenden a partir de los datos.

En esta sección se realiza un recordatorio de conceptos necesarios para
trabajar con probabilidades en el resto del texto.

\defineb
Una \emph{variable aleatoria} es una función medible
\(X:\Omega\rightarrow E\) donde \(\Omega\) es un espacio de probabilidad
y \(E\) un espacio medible. \definee
\defineb
El par \((\Omega, \Sigma)\) donde \(\Omega\) es un conjunto y \(\Sigma\)
una \(\sigma\)-álgebra sobre \(\Omega\) es un \emph{espacio medible}.
\definee
\defineb
Si \((\Omega, \mathcal{F})\) es un espacio medible y \(\mu\) es una
medida sobre \(\mathcal{F}\), entonces la terna
\((\Omega, \mathcal{F}, \mu)\) es un \emph{espacio de medida}. Si
además se verifica \(\mu(\Omega)=1\), entonces se trata de un
\emph{espacio de probabilidad}. \definee

Intuitivamente, una variable aleatoria representa una variable del
problema que puede tomar distintos valores, y la probabilidad con la que
se darán dichos valores puede ser descrita por una distribución de
probabilidad. Cuando notamos \(X:\Omega\rightarrow E\), interpretamos
que \(\Omega\) es el conjunto de todos los sucesos posibles, y los
estados que puede tomar la variable \(X\) vienen dados por su imagen,
\(X(\Omega)\subset E\). Se dice que \(X\) es \emph{discreta} si
\(X(\Omega)\) es numerable (incluyendo el caso finito), y es
\emph{continua} si \(X(\Omega)\) es no numerable.


\subsection{Notación de probabilidad}

En ocasiones preferiremos hablar de la probabilidad de que se dé un suceso, en lugar de la probabilidad de un valor concreto. Dado un espacio de probabilidad $(\Omega, \mathcal F, \mu)$ y una variable aleatoria $X:\Omega\rightarrow E$, notaremos
\[
  \Pr{X=x}=\mu(\{s\in\Omega:X(s)=x\})=\mu(X^{-1}(\{x\}))~.
\]

En general, podemos expresar de forma similar la probabilidad de una proposición lógica arbitraria sobre los valores de X. Si tenemos una condición $c$,
\[
  \Pr{X\mbox{ verifica }c}=\mu(\{s\in\Omega:X(s)\mbox{ verifica }c\})~.
\]


\subsection{Distribuciones de probabilidad}

\defineb
Una distribución de probabilidad sobre una variable discreta \(X\) se
describe mediante una \emph{función de probabilidad} (\emph{Probability
  Mass Function}, PMF) \(p:X(\Omega)\rightarrow [0,1]\), que se define como
\[
  p(x)=\Pr{X=x}~.
\]
\definee

\defineb
Una distribución de probabilidad sobre una variable continua \(X\) valuada en los números reales se
describe mediante una \emph{función de densidad} (\emph{Probability
  Density Function}, PDF) \(p:X(\Omega)\rightarrow [0,1]\), definida como
\[
  p(x)=\frac{d}{dx}\mu([-\infty,x])~,
\]
donde \(x\mapsto \mu([-\infty,x])\) se denomina \emph{función de distribución}.
\definee

Es común el abuso del lenguaje por el cual se nota a varias funciones de probabilidad de distintas variables aleatorias por la misma letra, pero no hay posibilidad de confusión ya que se evalúan en distintos valores.

\subsection{Distribuciones marginales}\label{marginal}

Cuando una distribución describe varias variables, puede interesar
conocer la distribución de un subconjunto de las mismas. Esta se
denomina \emph{distribución marginal}, y se consigue sumando o
integrando a lo largo de todos los valores de las variables que no están
en el subconjunto. Por ejemplo, si \(X\) e \(Y\) son variables
discretas, se tiene \[p(x) = \sum_{y\in Y(\Omega)}p(x, y)~.\] Si son
continuas, entonces se verifica
\[p(x) = \int\limits_{Y(\Omega)}p(x, y)dy~.\]

\subsection{Probabilidad condicionada}\label{probabilidad-condicionada}

En ocasiones es útil representar la probabilidad de un suceso
condicionado a la ocurrencia de otro. Para ello se utilizan
\emph{probabilidades condicionadas}, que se notan \(p(y|x)\) (donde
\(y\in Y(\Omega), x\in X(\Omega)\)) y se calculan mediante la siguiente
fórmula, suponiendo que \(p(x) > 0\):

\begin{equation}p(y|x)=\frac{p(y,x)}{p(x)}~.\label{eq:cond}\end{equation}

%\subsubsection{Encadenando probabilidades
%condicionadas}\label{encadenando-probabilidades-condicionadas}

Una distribución de probabilidad conjunta sobre varias variables se
puede descomponer como probabilidades condicionadas sobre una sola
variable:
\[p(x_1, \dots x_n) = p(x_1)\prod\limits_{i=2}^n p(x_i\mid x_1, \dots x_{i-1})~.\]

Esta expresión se deduce por inducción de la ecuación~\eqref{eq:cond}.

\subsection{Independencia e independencia
condicionada}\label{independencia-e-independencia-condicionada}

\defineb
Dos variables aleatorias, \(X\) e \(Y\), son \emph{independientes} si
la su probabilidad conjunta equivale al producto de sus probabilidades:
\[p(x,y)=p(x)p(y)\forall x\in X(\Omega),y\in Y(\Omega)~.\]
\definee

\defineb
Además, se dice que son \emph{condicionalmente independientes}
respecto a una variable \(Z\) si la distribución de probabilidad
condicionada se factoriza por \(X\) e \(Y\):
\[p(x,y|z)=p(x|z)p(y|z)\forall x\in X(\Omega),y\in Y(\Omega),z\in Z(\Omega)~.\]
\definee

\subsection{Momentos: esperanza, varianza y
covarianza}\label{momentos-esperanza-varianza-y-covarianza}

Para hablar de los momentos de variables aleatorias, nos centramos en las valuadas en los números reales. En el caso de una variable con codominio $\RR^{k}$, se pueden estudiar los momentos marginales de forma análoga.

\defineb
La \emph{esperanza} de una variable aleatoria \(X\) viene dada por las
expresiones siguientes, para variables discretas y continuas
respectivamente:
\[\mathrm E[X]=\sum_{x\in X(\Omega)}xp(x);\quad \mathrm E[X]=\int_{X(\Omega)}xp(x)dx~.\]
\definee

\note{Todos los momentos se toman respecto de una variable
aleatoria y una distribución de probabilidad asociada, por lo que la
notación correcta sería \(\mathrm E_{X\sim p}[X]\). Sin embargo, se omitirá
excepto para prevenir ambigüedades.}

Se puede definir la esperanza de una función \(f\) sobre los valores de
una variable aleatoria, del siguiente modo:
\[\mathrm E[f(X)]=\sum_{x\in X(\Omega)}f(x)p(x);\quad \mathrm E[f(X)]=\int_{X(\Omega)}f(x)p(x)dx~.\]

\defineb
La \emph{varianza} da idea acerca de cómo de diferentes entre sí son
los valores de una variables conforme se muestrean por su distribución
de probabilidad: \[\mathrm{Var}(X)=\mathrm E[(X-\mathrm E[X])^2]~.\]
\definee

\defineb
La \emph{covarianza} relaciona dos variables aleatorias, indicando la
medida en que están relacionadas linealmente y la escala de dichas
variables:
\[\mathrm{Cov}(X, Y)=\mathrm E[(X-\mathrm E[X])(Y-\mathrm E[Y])]~.\]
\definee

\defineb
Para un vector de variables aleatorias, \(X=(X_1, \dots X_n)\), la
\emph{matriz de covarianza} se define como una función matriz
\(n\times n\) dada por:
\[\mathrm{Cov}(X)_{i,j}=\mathrm{Cov}(X_i, X_j)~.\]
\definee

\section{Resultados de convergencia}\label{resultados-de-convergencia}

Ahora nos situamos en distribuciones de probabilidad sobre espacios vectoriales reales. En concreto, sobre $\RR^k$ para $k\geq 1$. Existen distintos conceptos de convergencia que podemos definir, aquí trabajaremos principalmente con la convergencia en probabilidad.

Sea \(d\) una distancia en \(\RR^k\) y sea
\(\{X_n:\Omega\rightarrow\RR^k\}\) una sucesión de variables aleatorias,
sea \(X:\Omega \rightarrow \RR^k\) una variable aleatoria.

\defineb
Se dice que \(X_n\) \emph{converge en probabilidad} a \(X\) si para cada
\(\varepsilon>0\) se tiene \(\Pr{d(X_n, X)>\varepsilon}\rightarrow 0\). Lo
denotamos \(X_n\pconv X\). \definee

\defineb
Se dice que \(X_n\) \emph{converge casi seguramente} a \(X\) si se da la
convergencia puntual en un conjunto de medida 1:
\[X_n\asconv X\Leftrightarrow \Pr{\lim_{n\rightarrow +\infty} d(X_n, X)=0}=1\]
\definee

Es un resultado conocido que \(X_n\pconv X\Rightarrow X_n\asconv X\).

\lemmab
\label{lm:convergencia-va} Si \(\{X_n\}\) es una sucesión de variables
aleatorias con varianza finita y se verifican las siguientes
condiciones:
\[\exists x\in \mathbb R:\lim_{m\rightarrow +\infty} \mathrm{E}[X_m]=x,\quad \lim_{m\rightarrow +\infty} \mathrm{Var}[X_m]=0,\]
entonces se tiene que \(X_m\pconv x\). \lemmae

\theob[Teorema de la aplicación continua]
\label{th:cont-map-conv} Sea \(\{X_n\}\) una sucesión de variables
aleatorias y \(X\) una variable aleatoria, valuadas en un espacio
medible \(E\). Sea \(g:E\rightarrow F\) con \(F\) otro espacio medible.
Entonces, si \(g\) es continua casi por doquier, se tiene:

\begin{gather*}
  X_n\pconv X \Rightarrow g(X_n)\pconv g(X),\\
  X_n\asconv X \Rightarrow g(X_n)\asconv g(X).
\end{gather*}

\theoe

%\section{\textasciitilde{}Herramientas de inferencia estadística
%(?)\textasciitilde{}}\label{herramientas-de-inferencia-estaduxedstica}

%\subsection{Estimadores
%máximo-verosímiles}\label{estimadores-muxe1ximo-verosuxedmiles}

\section{La maldición de la
dimensionalidad}\label{sec:dim-curse}

Vamos a aplicar los resultados teóricos anteriores para estudiar una propiedad interesante que determinará uno de los problemas tratados en el campo del aprendizaje automático (\autoref{sec:red-dim}). Supongamos que contamos con una muestra de datos, en forma de subconjunto finito de $\RR^{n}$. Nos podemos plantear qué efecto tiene el tamaño de $n$, en ocasiones denominado \emph{dimensionalidad}, sobre nuestra capacidad para extraer información útil de los datos. 

Algunos de los algoritmos más usuales utilizan distancias para medir similitudes entre los datos. Veremos que, conforme $n$ crece, las distancias usuales pierden significado, en el
sentido de que el punto más lejano y el más cercano a uno dado están a
distancias similares. Este hecho se suele denominar la maldición de la
alta dimensionalidad (del inglés \emph{curse of high dimensionality}). Una
formalización se encuentra en \textcite{beyer1999}, y se expone a
continuación:

\theob
\label{th:dim-curse}
Sea \(\{F_{m}\}_{m\in\NN}\) una sucesión de distribuciones de
probabilidad, \(n\in \mathbb N\) y \(p\in\mathbb R^+\) fijos. Para cada
\(m\in\NN\) sean \(X_{m1},\dots,X_{mn}\sim F_m\) muestras independientes
e idénticamente distribuidas. Supongamos que tenemos una función
\(d_m:\mathrm{Dom}(F_m)\rightarrow \mathbb R^+_0\) y llamamos

\begin{align*}
  \mathrm{DMIN}_{m}&=\min\{d_m(X_{mi}):i=1,\dots,n\},\\
  \mathrm{DMAX}_{m}&=\max\{d_m(X_{mi}):i=1,\dots,n\}.
\end{align*}

Entonces, si
\(\lim_{m\rightarrow +\infty}\Var\left[\frac{d_m(X_{m1})^p}{E[d_m(X_{m1})^p]}\right]=0\)
se tiene que, para cada \(\varepsilon > 0\),
\[\lim_{m\rightarrow +\infty}\Pr{\mathrm{DMAX}_m\leq (1+\varepsilon) \mathrm{DMIN}_m}=1.\]
\proofb

Puesto que las muestras \(X_{mi}\) son idénticamente distribuidas,
tienen la misma esperanza, y funciones de las mismas también comparten
esperanza. Así, llamamos \(\mu_m = \E[d_m(X_{mi})^p]\) y sea
\(V_m =\frac{d_m(X_{m1})^p}{\mu_m}\).

Veamos que \(V_m\pconv 1\): primero, tenemos que
\(\E[V_m] = \frac{\mu_m}{\mu_m} = 1\), y como consecuencia
\(\lim_{m\rightarrow +\infty}\E[V_m] = 1\). Por hipótesis,
\(\lim_{m\rightarrow +\infty}\Var[V_m] = 1\), y usando el
\autoref{lm:convergencia-va} deducimos que \(V_m\pconv 1\).

Ahora, definimos la variable aleatoria
\[Y_m=\left(\frac{d_m(X_{m1})^p}{\mu_m}, \dots, \frac{d_m(X_{mn})^p}{\mu_m}\right).\]
Como cada componente del vector \(Y_m\) es idénticamente distribuida a
\(V_m\), se tiene que \(Y_m\pconv (1, \dots, 1)\). Como \(\min\) y
\(\max\) (que dan la componente mínima y máxima del vector,
respectivamente) son funciones continuas, podemos utilizar el \autoref{th:cont-map-conv} para obtener que
\(\min(Y_m)\pconv \min\{1, \dots, 1\} = 1\) y \(\max(Y_m)\pconv 1\).

Notemos ahora que
\(\mathrm{DMIN}_m= \min\{\mu_m Y_m(i):i=1,\dots,n\}=\mu_m \min(Y_m)\) y
de igual forma \(\mathrm{DMAX}_m=\mu_m \max(Y_m)\). Así,
\[ \frac{\mathrm{DMAX}_m}{\mathrm{DMIN}_m}=\frac{\mu_m \max(Y_m)}{\mu_m \min(Y_m)}=\frac{\max(Y_m)}{\max(Y_m)}\pconv \frac 1 1= 1.\]

Por definición de convergencia en probabilidad, para cada
\(\varepsilon>0\) se tiene

\begin{equation}
  \label{eq:conv-dmax-dmin}
  \lim_{m\rightarrow +\infty} \Pr{\left\lvert \frac{\mathrm{DMAX}_m}{\mathrm{DMIN}_m} - 1 \right\rvert\leq\varepsilon} = 1,
  \end{equation}
y usando que \(\Pr{\mathrm{DMAX}_m \geq \mathrm{DMIN}_m}=1\),
\begin{gather*}
  \Pr{\left\lvert \frac{\mathrm{DMAX}_m}{\mathrm{DMIN}_m} - 1 \right\rvert\leq\varepsilon}=
\Pr{\frac{\mathrm{DMAX}_m}{\mathrm{DMIN}_m} - 1 \leq\varepsilon}=\\=
\Pr{\mathrm{DMAX}_m\leq (1+ \varepsilon)\mathrm{DMIN}_m },
\end{gather*}
luego el límite \eqref{eq:conv-dmax-dmin} es el que queríamos demostrar.
\proofe
\theoe

Nótese que este resultado es más general de lo que necesitamos, usando
cualquier función valuada no negativa \(d_m\) que podemos interpretar
como la distancia a un punto fijo. Como caso particular, en
\textcite{aggarwal2001} se prueba el resultado para la distancia
asociada a la norma \(L_p\). Además, no menciona realmente la
dimensionalidad, que se puede interpretar como un caso particular de la
cantidad \(m\) del teorema.

Por otro lado, requiere de una condición que no necesariamente se dará
en todos los escenarios,
\(\lim_{m\rightarrow +\infty}\Var\left[\frac{d_m(X_{m1})^p}{E[d_m(X_{m1})^p]}\right]=0\).
Un análisis de las situaciones en que el resultado es aplicable se
encuentra de nuevo en \textcite{beyer1999}. Esencialmente, es suficiente
que las distribuciones de los datos sean independientes e idénticamente
distribuidas a lo largo de todas las dimensiones, y los momentos
convenientes sean finitos. También se aportan varios ejemplos donde no se
da la independencia y sí se verifican las condiciones del teorema.

%\include{Chapters/Chapter0A}
%********************************************************************
% Other Stuff in the Back
%*******************************************************
\cleardoublepage\include{meta/Bibliography}
%\cleardoublepage\include{FrontBackmatter/Declaration}
%\cleardoublepage\include{FrontBackmatter/Colophon}
% ********************************************************************
% Game Over: Restore, Restart, or Quit?
%*******************************************************
\end{document}
% ********************************************************************

%*******************************************************
% Abstract
%*******************************************************
%\renewcommand{\abstractname}{Abstract}
\pdfbookmark[1]{Resumen}{Resumen}

\chapter*{Resumen}
\textcolor{red}{\textbf{Redactar esto}}

La clasificación es una tarea del aprendizaje automático muy estudiada en la actualidad, debido al
incremento notable de la información que se genera y se recopila, y se pretende utilizar para
predecir la clase de nuevos ejemplos. En multitud de casos no es suficiente realizar un
aprendizaje sobre la información, ya que esta puede presentar ciertos obstáculos (ruido, valores
ausentes, desbalanceo de clases), que hace preciso el tratamiento previo mediante técnicas de
preprocesamiento de datos.

El campo de Deep Learning engloba un conjunto de técnicas que tratan con el problema de la
complejidad que representan los datos mediante amplias jerarquías de representaciones más simples
de los datos. Dichas técnicas se basan en los conceptos de neurona artificial y redes neuronales,
que se fundamentan en la teoría de la información y teoría de probabilidad, entre otros ámbitos
matemáticos. Además, algunas de ellas tienen aplicaciones en el preprocesamiento de datos, por
ejemplo mediante la extracción de nuevas características a partir de las originales, lo que permite
reducir la dimensionalidad del conjunto de datos. Ejemplos de estos métodos son los autoencoders
y las Restricted Boltzmann Machines. En la actualidad, existen distintas herramientas que
realizan aprendizaje mediante las técnicas mencionadas, pero no facilitan el estudio visual y
exploratorio de las representaciones internas aprendidas.

El presente trabajo fin de grado tiene como objetivo el estudio de los fundamentos matemáticos de
las técnicas de Deep Learning orientadas a preprocesamiento y reducción de la dimensionalidad
más relevantes y el desarrollo de un paquete software que las recopile y permita el análisis visual y
exploratorio por distintos métodos. De esta forma se compensará la falta de interpretabilidad de las
representaciones aprendidas por las técnicas, y se compondrá una herramienta que facilite dicho
análisis. Asimismo, se realizará un estudio comparativo del rendimiento de las técnicas relacionado
con los resultados generados por el software desarrollado, con el objetivo de valorar la medida en
que ayudan a conocer el comportamiento de cada una de las técnicas.

\begin{otherlanguage}{american}
\pdfbookmark[1]{Abstract}{Abstract}
\chapter*{Abstract}
Wheeeew
\end{otherlanguage}
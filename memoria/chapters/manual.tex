\section{Documentación para el paquete Ruta en la versión 0.2.0}

\subsection{Predecir salidas para modelos entrenados y nuevos datos}

\paragraph{Descripción}
Se deben entregar al menos un argumento de modelo y uno de tarea. La tarea puede contener los mismos datos de entrada que se utilizaron para entrenar el modelo, o datos nuevos con la misma estructura. El resto de argumentos serán facilitados a la función interna de predicción.

\paragraph{Uso}
~

\begin{lstlisting}
## S3 method for class 'rutaModel'
predict(object, ...)\end{lstlisting}

\paragraph{Argumentos}
\begin{itemize}
\item \emph{object}	Un objeto \texttt{``rutaModel''}.
\item \emph{...}	Parámetros específicos para la función de predicción interna.
\end{itemize}

\paragraph{Valor}
Una matriz conteniendo predicciones para cada instancia de la tarea dada.

\subsection{Obtener las características internas de un modelo de autoencoder entrenado}

\paragraph{Descripción}
Extrae las características de la capa de codificación de un autoencoder previamente entrenado.

\paragraph{Uso}
~

\begin{lstlisting}
ruta.deepFeatures(model, task, ...)
\end{lstlisting}

\paragraph{Argumentos}
\begin{itemize}
\item \emph{model}	Un objeto \texttt{``rutaModel''} de un modelo de autoencoder.
\item \emph{task}	Un objeto \texttt{``rutaTask''}.
\item \emph{...}	Parámetros específicos para la función de predicción interna.
\end{itemize}

\paragraph{Valor}
Una matriz conteniendo la codificación para cada instancia de la tarea dada.


\subsection{Obtener los pesos de cualquier capa de un modelo entrenado}

\paragraph{Descripción}
Extrae los pesos de los parámetros fijados al entrenar un modelo.

\paragraph{Uso}
~

\begin{lstlisting}
ruta.getWeights(model, layer)
\end{lstlisting}

\paragraph{Argumentos}
\begin{itemize}
\item \emph{model}	Un objeto \texttt{``rutaModel''} de un modelo de autoencoder.
\item \emph{layer}	Un entero indicando el índice de la capa de la que se extraerán los pesos.
\end{itemize}

\paragraph{Valor}
Una matriz conteniendo los pesos de la capa dada.


\subsection{Obtener las salidas de cualquier capa de un modelo entrenado}

\paragraph{Descripción}
Extrae las salidas de la capa indicada a partir de un modelo entrenado y datos nuevos

\paragraph{Uso}
~

\begin{lstlisting}
ruta.layerOutputs(model, task, layerInput = 1, layerOutput, ...)
\end{lstlisting}

\paragraph{Argumentos}
\begin{itemize}
\item \emph{model}	Un objeto \texttt{``rutaModel''} de un modelo de autoencoder o RBM.
\item \emph{task}	Un objeto \texttt{``rutaTask''} conteniendo los datos que deben propagarse por la red.
\item \emph{layerInput}	Un entero indicando el índice de la capa en la que se inyectarán los datos. En autoencoders se inyectan siempre en la primera capa.
\item \emph{layerOutput}	Un entero indicando el índice de la capa que se pretende extraer.
  \item \emph{...} Parámetros específicos para la función de predicción interna.
\end{itemize}

\paragraph{Valor}
Una matriz conteniendo salidas para cada instancia de la capa dada.


\subsection{Representar un algoritmo de aprendizaje}

\paragraph{Descripción}
Crea un objeto que representa un algoritmo de aprendizaje.

\paragraph{Uso}
~

\begin{lstlisting}
ruta.makeLearner(cl, id = cl, ...)
\end{lstlisting}

\paragraph{Argumentos}
\begin{itemize}
\item \emph{cl}	Una cadena de caracteres indicando el tipo de algoritmo (\texttt{``autoencoder''} o \texttt{``rbm''}).
\item \emph{id}	Una cadena de caracteres opcional indicando un nombre para el objeto.
  \item \emph{...} Parámetros adicionales para el objeto, específicos al algoritmo requerido.
\end{itemize}

\paragraph{Valor}
Un objeto para el algoritmo de aprendizaje que almacena los parámetros provistos.

\paragraph{Ejemplos}
~

\begin{lstlisting}
ruta.makeLearner("rbm", hidden = 4, activation = "bin")
ruta.makeLearner("autoencoder", "ae1",
                 hidden = c(4, 2, 4),
                 activation = "relu")
               \end{lstlisting}


\subsection{Representar una tarea de aprendizaje}

\paragraph{Descripción}
Crea un objeto que representa una tarea de aprendizaje.

\paragraph{Uso}
~

\begin{lstlisting}
ruta.makeTask(id, data, cl = NULL)
\end{lstlisting}

\paragraph{Argumentos}
\begin{itemize}
\item \emph{id}	Una cadena de caracteres opcional indicando un nombre para el objeto.
\item \emph{data}	Un conjunto de datos, será convertido a un \emph{data.frame}.
\item \emph{cl}	El índice de la columna correspondiente a la clase, o \texttt{NULL} si el conjunto no incluye clase.
\end{itemize}

\paragraph{Valor}
Un objeto genérico para la tarea de aprendizaje que almacena los datos dados.

\subsection{Representar una tarea de aprendizaje no supervisado}

\paragraph{Descripción}
Crea un objeto que representa una tarea de aprendizaje no supervisado.

\paragraph{Uso}
~

\begin{lstlisting}
ruta.makeUnsupervisedTask(id, data, cl = NULL)
\end{lstlisting}

\paragraph{Argumentos}
\begin{itemize}
\item \emph{id}	Una cadena de caracteres opcional indicando un nombre para el objeto.
\item \emph{data}	Un conjunto de datos, será convertido a un \emph{data.frame}.
\item \emph{cl}	El índice de la columna correspondiente a la clase, o \texttt{NULL} si el conjunto no incluye clase.
\end{itemize}

\paragraph{Valor}
Un objeto para la tarea de aprendizaje no supervisado que almacena los datos dados y los parámetros.

\paragraph{Ejemplo}
~

\begin{lstlisting}
data(iris)
irisTask <- ruta.makeUnsupervisedTask(
              "iris",
              data = iris,
              cl = 5)
\end{lstlisting}

\subsection{Pre-entrenamiento de autoencoders}

\paragraph{Descripción}
Obtener los pesos iniciales para un autoencoder mediante una pila de RBMs.

\paragraph{Uso}
~

\begin{lstlisting}
ruta.pretrain(x, task, epochs = 10, ...)
\end{lstlisting}

\paragraph{Argumentos}
\begin{itemize}
\item \emph{x}	Un objeto \texttt{``rutaAutoencoder''}.
\item \emph{task}	Un objeto \texttt{``rutaTask''}.
\item \emph{epochs}	El número de épocas para el proceso de entrenamiento de cada RBM.
\end{itemize}

\paragraph{Valor}
Una lista de matrices de pesos iniciales para facilitar como el argumento \texttt{initial.args} en una llamada a \texttt{train} con un objeto de clase \texttt{``rutaAutoencoder''}.


\subsection{Entrenamiento de modelos}

\paragraph{Descripción}
Generar un modelo entrenado para un objeto asociado 

\paragraph{Uso}
~

\begin{lstlisting}
train(x, task, ...)
\end{lstlisting}

\paragraph{Argumentos}
\begin{itemize}
\item \emph{x}	Un objeto representando un algoritmo de aprendizaje de clase \texttt{``rutaAutoencoder''} o \texttt{``rutaRBM''}.
\item \emph{task}	Un objeto \texttt{``rutaTask''}.
\item \emph{...} Resto de parámetros según tipo de algoritmo.

  En el caso de un autoencoder:
  \begin{itemize}
  \item \emph{epochs}	El número de iteraciones sobre el algoritmo de optimización.
  \item \emph{optimizer} El nombre del optimizador a usar (\texttt{``sgd''}, \texttt{``adagrad''}, \texttt{``rmsprop''}, \texttt{``adam''} o \texttt{``adadelta''}).
  \item \emph{eval.metric} Medida de evaluación (por defecto, el error cuadrático medio).
  \item \emph{initial.args} Lista de argumentos iniciales (pesos obtenidos por \texttt{ruta.pretrain}).
  \item \emph{initializer.scale} Escala de la distribución uniforme para inicializar pesos.
  \item \emph{...} Parámetros adicionales para el optimizador.
  \end{itemize}

  En el caso de una RBM:
  \begin{itemize}
  \item \emph{numepochs} El número de iteraciones para el entrenamiento.
  \end{itemize}
\end{itemize}

\paragraph{Valor}
Un objeto de clase \texttt{``rutaModel''} que contiene el modelo entrenado.

\paragraph{Ejemplo}
~

\begin{lstlisting}
data(iris)
irisTask <- ruta.makeUnsupervisedTask(
              "iris",
              data = iris,
              cl = 5)

ae <- ruta.makeLearner(
        "autoencoder",
        hidden = c(4, 2, 4),
        activation = "leaky")
model <- train(
           ae,
           task,
           epochs = 80,
           optimizer = "adagrad",
           learning.rate = 0.02,
           initializer.scale = 2)         
\end{lstlisting}

\section{Documentación para el paquete Rutavis en la versión 0.2.0}

\subsection{Representar la codificación de un modelo entrenado}

\paragraph{Descripción}
Obtener y representar la capa interna de un modelo entrenado con unos datos de entrada.

\paragraph{Uso}
~

\begin{lstlisting}
## S3 method for class 'rutaModel'
plot(model, task, ...)
\end{lstlisting}

\paragraph{Argumentos}
\begin{itemize}
\item \emph{model}	Un objeto \texttt{``rutaModel''} entrenado mediante un autoencoder.
\item \emph{task}	Un objeto \texttt{``rutaTask''} con los datos de entrada.
\item \emph{...}	Parámetros adicionales para la función de representación gráfica.
\end{itemize}

\paragraph{Valor}
Genera un gráfico de Plotly y lo muestra en un panel interactivo.


\subsection{Lanzar la interfaz de usuario web}

\paragraph{Descripción}
Carga la interfaz gráfica de usuario web y abre una nueva pestaña de navegador para mostrarla.

\paragraph{Uso}
~

\begin{lstlisting}
ruta.gui()
\end{lstlisting}

\paragraph{Valor}
No devuelve nada.